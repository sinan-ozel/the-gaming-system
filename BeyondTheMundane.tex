\documentclass{LegrandOrangeTufteBook}

% Add some TODO notes for the writer's own use.
\usepackage[disable]{todonotes}

% This package combines longtable and tabularx
\usepackage{xltabular}

% Add the highlighted notes that appear when the reader hovers
\usepackage{acro}
\usepackage{pdfcomment}

% Get some nice pre-defined colors.
% Start here: https://ctan.math.ca/tex-archive/macros/latex/contrib/xkcdcolors/xkcdcolors-manual.pdf
% See the git page: https://github.com/Rmano/xkcdcolors
% See the full list here: https://xkcd.com/color/rgb/
% See here for the story: https://blog.xkcd.com/2010/05/03/color-survey-results/
\usepackage{xkcdcolors}

% Define coloring related to terms
% I am using "Contrasting Palette 1 & 2" from https://venngage.com/tools/accessible-color-palette-generator
\definecolor{ocre}{RGB}{196, 70, 1} % Define the color used for highlighting throughout the boo

\definecolor{colorCoreCompetency}{RGB}{91, 163, 0}
\newcommand{\termCore}[1]{\textcolor{colorCoreCompetency}{#1}}

\definecolor{colorTerm}{RGB}{196, 70, 1}
\newcommand{\term}[1]{\textcolor{colorTerm}{#1}}

\newcommand{\termClass}[1]{\textcolor{xkcdSpearmint}{#1}}

% Add the nice side tabs on the tables.
% Usage: Put into the rightmost cell in a tabularx environment.
% Example: \sideTab{LightBlue}{Modern}
\newcommand{\sideTab}[2]{\cellcolor{#1} \rotatebox[origin=l]{270}{#2}}

% Add color to table cells
% See here: https://tex.stackexchange.com/questions/50349/color-only-a-cell-of-a-table
\usepackage{colortbl}

% Use this package to break up a list into multiple columns
\usepackage{multicol}

% Style the bullets in itemized lists
% https://tex.stackexchange.com/questions/42805/what-are-original-itemize-bullet-definitions
\renewcommand\labelitemi{\textbullet}
\renewcommand\labelitemii{\normalfont\bfseries \textendash}
\renewcommand\labelitemiii{\textasteriskcentered}
\renewcommand\labelitemiv{\textperiodcentered}


% Make it easy to create the core competencies of a character sheet
\newcommand{\coreCompetencyTable}[8]{
	\begin{center}
	\resizebox{\columnwidth}{!}{
		\begin{tabular}{lr|clr|}
			\multicolumn{2}{c}{Mental}	&&	\multicolumn{2}{c}{Physical} \\
			\cline{1-2} \cline{4-5}
			Reasoning & #1				&& 	Coordination & #3 \\
			Situational Awareness & #2	&& 	Constitution & #4 \\
			\multicolumn{4}{c}{} \\
			\multicolumn{2}{c}{Social}	&&	\multicolumn{2}{c}{Innate} \\
			\cline{1-2} \cline{4-5}
			Communication & #5 			&&	Focus & #7 \\
			Social Awareness & #6 		&&	Creativity & #8 	\\
		\end{tabular}
	}
	\end{center}
}

\newcommand{\hover}[2]{
	\pdfmarkupcomment[markup=Highlight,disable=false,color=LightYellow]{#1}{#2}
}



% End of the common part of the preamble.

\begin{document}

\chapterimage{image/transmundane.jpg}
\chapterspaceabove{6.75cm}
\chapterspacebelow{11.25cm}


\chapter*{Going Beyond The Mundane}

\term{Mundane} characters are for creating characters that you would see in real life.
You are not limited to these characters for gameplay.
You can create NPCs and PCs that go beyond the \term{mundane}.
There are three game mechanics for creating \term{transmundane} characters.

\begin{enumerate}
    \item If you create a character (or advance them) with one of their \term{core competencies} above 2d6, that is going beyond the mundane. They are going to have an easier time passing challenges that are difficult for regular charactersm as their dice increases to d6 + d8, d6 + d10 or 2d8, and so on.
    \item If you add a third die to one of the \term{core competencies}, this also takes the character beyond the mundane. This is more powerful than the previous option, but is also more penalizing, because it increases the probability of rolling a double one, a \term(misfortune).
    \item Finally, you can add spellcasting or supernatural powers to the character.
\end{enumerate}

Any combination of these are possible. The following sections describe these in further detail.

\section*{Increasing Core Competencies Beyond 2d6}

This is the basic way of creating a character beyond the \term{mundane}.
\marginpar{
    \footnotesize
    \begin{tabular}{rl}
\multicolumn{2}{c}{At least one die}\\
d8 & Hero \\
d10 & Superhero \\
d12 & Legendary \\
d20 & Demi-god \\
\end{tabular}
}

A typical talented adult will have a 2d6 in their best \term{core competency}.
A slightly more powerful character can have d6 + d8.
Going even further beyond, either the higher die or the lower die can be increased.\\

This type of \term{transmundane} character is fit for superhero campaigns,
and for creating enemies that are immensely powerful, such as ancient vampires.

\todo{Add an example, and also show that the character's mundane level changes.}

\todo{What are the rules for character advancement?}
\todo{What is the significance of increasing the larger die rather than the smaller?}

\section*{Adding a Third Die}

This is how you create tragic characters, PC and NPC alike.
This is a powerful way of making the character powerful, but with a cost.
The character will have an easier time passing regular challenges,
however, their probability of rolling a double one, therefore a \term{misfortune}
will be increased.\\

This is for roleplaying potentially dangerous or forbidden magic, alien technology and contracts with otherworldly entities.

\section*{Spells and Powers}

How many times have you come across a flying city, an ancient curse, or an
undead army? Typically, these are elements of fantasy, but there is no
way for players to access these situations. This chapter aims to fix
this by giving a framework for creating spells.\\

Depending on the effect, use either one of the following five frameworks.
\begin{enumerate}
	\item Matter Framework
	\item Life Framework
	\item Mind Framework
	\item Energy Framework
	\item Narrative Framework
\end{enumerate}

If the spell or power creates, changes, or directly breaks down objects or materials, use the matter framework.
This include shape changes, chemical changes \& molecular changes.\\

If the spell or power creates, changes, or directly breaks down living beings with a material existence, use the life framework.\\

If the spell or power impacts the thought and emotions of a sentient being, use the mind framework.\\

For everything else, use the energy framework.\\

\subsection*{Energy Spells \& Powers}

To keep energy spells consistent,
I use their energy requirement to determine their \term{difficulty level}.
\todo{This may create difficulties during game play if the characters want to create spells or powers. How to resolve this if a character wants to create a hitherto unused effect, such as sorcerors?}

Power Requirement: Use the number of digits in the power requirement as the addition to the \term{difficulty level}.\\
Duration: Use the table on the side.\footnote{This is also based on the number of digits in terms of seconds of duration.}
\marginpar{
	\footnotesize
	\begin{tabularx}{\marginparwidth}{rX}
\multicolumn{2}{c}{Duration Table} \\
\hline
+0 & Instantaneous or one second \\
+1 & 1 minute \\
+2 & 10 minutes \\
+3 & 1 hour \\
+4 & 1 day \\
+5 & 1 week \\
+6 & 1 month \\
+7 & 1 year \\
+8 & 1 decade \\
+9 & 1 century \\
+10 & 1 millenium \\
+11 & 10 millenia \\
\end{tabularx}
}

\marginpar{
	\footnotesize
	\begin{tabularx}{\marginparwidth}{rXX}
\multicolumn{3}{c}{Distance \& Difficulty Level} \\
DL Increase & Metric & Imperial \\
\hline
+1 & 1 m & 1 yd \\
+2 & 10 m & 10 yd \\
+3 & 100 m & 100 yd \\
+4 & 1 km & 1 mi \\
+5 & 10 km & 10 mi \\
+6 & 100 km & 100 mi \\
% +7 & 1000 km & 1000 mi \\
% +8 & 10000 km & 10000 mi \\
\multicolumn{3}{c}{...} \\
+12 & \multicolumn{2}{c}{1 AU}  \\
\end{tabularx}
}


\subsubsection*{Light}

Creating a lightbulb worth of light right above the character's head (+1) for just one second (+1) requires 5 Wt of power +1.\\

\subsubsection*{Lightning}
Let's consider a spell or power that can potentially harm people.
% TODO: Continue

\subsubsection*{Fire}

\subsubsection*{Heat}

\subsubsection*{Wind}

\subsubsection*{Physical Force}

\subsection*{Matter Spells \& Powers}

To keep

\subsection*{Mind Spells \& Powers}

\subsection*{Impacting the Dice - Meta Spells \& Powers}

\subsection*{Combining Multiple Impacts}

\todo{
    Think about this.
    This is probably an extended challenge.
    For example, to create a clay golem, you might first transform the clay to being soft.
    Then you can create a mind.
    What about ice shards? You could create water, cool it with energy, and then move it with force... How to do that in combat?
}

\todo{Add a table of spells \& powers}

\section*{Transmundane Classes}

Classes are only for \term{transmundane} characters.\\

To balance gameplay for \term{transmundane} player characters,
sum the \term{difficulty level} of their \term{transmundane} powers.
This is going to be their \term{transmundane level}.
\footnote{
    For powers and spells whose \term{difficulty level} can go up indefinitely, choose the highest possible roll that the character can pass on their own.
    For instance, if the character rolls \termCore{Focus}
}
\todo{
    Finish this thought.
}\\

\subsection*{Wizards}

As a player roleplaying \termClass{wizard}, you can either choose your spells from the list below,
or create your own spells by doing research (in real life), and convincing the party \emph{and} the GM
that the difficulty level is correct.

\subsection*{Mages}

\subsection*{Druids}

\subsection*{Bards}

\subsection*{Demonologists \& Summoners}

\subsection*{Witches \& Warlocks}

\subsection*{Sorcerors \& Sorceresses}

\subsection*{Vampires}

\subsection*{Werewolves}

\subsection*{Superheroes}

\subsection*{Mediums}

\subsection*{Necromancers}

\subsection*{Clairvoyants}

\subsection*{Transmundane Healers}

\subsection*{Waterbenders}

\subsection*{Firebenders \& Elemental Magic}

\end{document}

