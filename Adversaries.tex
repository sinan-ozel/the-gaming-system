\documentclass{LegrandOrangeTufteBook}

% Add some TODO notes for the writer's own use.
\usepackage[disable]{todonotes}

% This package combines longtable and tabularx
\usepackage{xltabular}

% Add the highlighted notes that appear when the reader hovers
\usepackage{acro}
\usepackage{pdfcomment}

% Get some nice pre-defined colors.
% Start here: https://ctan.math.ca/tex-archive/macros/latex/contrib/xkcdcolors/xkcdcolors-manual.pdf
% See the git page: https://github.com/Rmano/xkcdcolors
% See the full list here: https://xkcd.com/color/rgb/
% See here for the story: https://blog.xkcd.com/2010/05/03/color-survey-results/
\usepackage{xkcdcolors}

% Define coloring related to terms
% I am using "Contrasting Palette 1 & 2" from https://venngage.com/tools/accessible-color-palette-generator
\definecolor{ocre}{RGB}{196, 70, 1} % Define the color used for highlighting throughout the boo

\definecolor{colorCoreCompetency}{RGB}{91, 163, 0}
\newcommand{\termCore}[1]{\textcolor{colorCoreCompetency}{#1}}

\definecolor{colorTerm}{RGB}{196, 70, 1}
\newcommand{\term}[1]{\textcolor{colorTerm}{#1}}

\newcommand{\termClass}[1]{\textcolor{xkcdBlueGreen}{#1}}

% Add the nice side tabs on the tables.
% Usage: Put into the rightmost cell in a tabularx environment.
% Example: \sideTab{LightBlue}{Modern}
\newcommand{\sideTab}[2]{\cellcolor{#1} \rotatebox[origin=l]{270}{#2}}

% Add color to table cells
% See here: https://tex.stackexchange.com/questions/50349/color-only-a-cell-of-a-table
\usepackage{colortbl}

% Use this package to break up a list into multiple columns
\usepackage{multicol}

% Style the bullets in itemized lists
% https://tex.stackexchange.com/questions/42805/what-are-original-itemize-bullet-definitions
\renewcommand\labelitemi{\textbullet}
\renewcommand\labelitemii{\normalfont\bfseries \textendash}
\renewcommand\labelitemiii{\textasteriskcentered}
\renewcommand\labelitemiv{\textperiodcentered}


% Make it easy to create the core competencies of a character sheet
\newcommand{\coreCompetencyTable}[8]{
	\begin{center}
	\resizebox{\columnwidth}{!}{
		\begin{tabular}{lr|clr|}
			\multicolumn{2}{c}{Mental}	&&	\multicolumn{2}{c}{Physical} \\
			\cline{1-2} \cline{4-5}
			Reasoning & #1				&& 	Coordination & #3 \\
			Situational Awareness & #2	&& 	Constitution & #4 \\
			\multicolumn{4}{c}{} \\
			\multicolumn{2}{c}{Social}	&&	\multicolumn{2}{c}{Innate} \\
			\cline{1-2} \cline{4-5}
			Communication & #5 			&&	Focus & #7 \\
			Social Awareness & #6 		&&	Creativity & #8 	\\
		\end{tabular}
	}
	\end{center}
}

\newcommand{\hover}[2]{
	\pdfmarkupcomment[markup=Highlight,disable=false,color=LightYellow]{#1}{#2}
}



% End of the common part of the preamble.

% TODO: Find a good image
\chapterimage{image/dice.jpg}
\chapterspaceabove{6.75cm}
\chapterspacebelow{11.25cm}


\chapter*{Adversaries}

\section*{Mundane}

\subsection*{Fire}

Fire is a beast: it has no reasoning or situational awareness, but it has a strength.
\begin{center}
	\resizebox{\columnwidth}{!}{
		\begin{tabular}{lr}
            \multicolumn{2}{c}{A Small Fire}
			Blaze & 1d4 \\
		\end{tabular}
	}
\end{center}

Each round, fire rolls its \termCore{Strength}. Every time it rolls 4, it spreads into an adjacent tile.
Once it goes over 4 tiles in either direction, its \termCore{Strength} increases.
The maximum number of tiles in a row in each direction determines one of the dice in the \termCore{Strength} roll, rounded up.
\todo{Add an image using tikz.}\\

\todo{Describe how to douze the fire.}
\todo{Describe what happens if the fire rolls a double-one.}

\todo{Link to the fire elemental subsection}

\section*{Beyond the Mundane}

\subsection*{Fire Elemental}
\label{subsec:fire_elemental}
\todo{Write the Fire Elemental. It needs to have Reasoning as a core competence.}

\end{document}
