\chapterimage{image/dice.jpg}
\chapterspaceabove{6.75cm}
\chapterspacebelow{11.25cm}


\chapter{Challenges}

\begin{emphasisParagraph}
	For basic challenges, the actor rolls a \term{core capability}.
	A \term{physiology} or \term{expertise} bonus is added.
	The result of the roll has to be greater than the \term{difficulty level (DL)} to succeed.
\end{emphasisParagraph}
\todo{Does the success level matter? Think on it.}

\section{The Basic Mechanic}
For basic challenges, the actor rolls a \term{core capability}..
A \term{physiology} or \term{expertise} bonus is added.
The result of the roll has to be greater than the \term{difficulty level (DL)} to succeed.\\

\paragraph*{The Bonus Limit}
The bonus cannot be greater than the maximum roll of the \term{core capability},
unless an \term{item}, \term{spell} or \term{power} specifically allows this.
For example, if the roll is 2d6, the bonus can be at most +12. If the roll is d4 + d6,
the bonus cannot be more than 4 + 6 = 10.\\

\paragraph*{Ties}
If the result of a roll, with the bonus, is \emph{equal} to the \acronymDL,
then this is a tie.

\paragraph*{The Double One}

If the roll has two 1's, this is a \term{double one}.
This is the GM's opportunity to introduce a misfortune to the game:
The actor, either the player or an NPC, may still succeed, thanks to their bonus.
However, circumstances outside their control or a tragic character flaw creates a misfortune.
This rule applies to both players and NPCs. \\

Interpret the \term{double one} as independent of whether the character
succeeds, fails, or ties. A \term{double one} is a misfortune that will
impact for the rest of the rest, and it is something happening not
because the character was not good enough, but because circumstances were
beyond their control.\\


In the cases where three dice are rolled,
a \term{double one} is enough to trigger a misfortune.
This mechanic increases the probability of a misfortune
whenever a third die is added. The purpose is to create
opportunities to roleplay immense power that comes at a
cost.\\
\todo{Add some NPC and PC examples.}

In the cases where only one die is rolled,
a \term{double one} is not possible. This is intentional:
These characters may be underdogs when $d4$ or $d6$ is the roll.
Consider this to be beginner's luck, or as a mechanic to protect the
underdog.\\

\paragraph*{Guideline on Deciding the Difficulty Level}

Imagine a talented character (2d6 in their \term{core capability}) with 20 years of experience.
This character's roll is 2d6 + 12 without any additional bonuses.
This means that they are approximately 50\% likely to overcome a challenge with \acronymDL 18.
It also means that they can overcome a challenge with \acronymDL 24 with a lot of luck - only 1/36 of the time.\\

\marginpar{
	\footnotesize
	\begin{tabularx}{\marginparwidth}{rX}
\multicolumn{2}{c}{Difficulty Levels} \\
3 & Teenagers fail half the time \\
6 & A teenager might fail, but an adult should get it right. \\
9 & An adult should succeed almost all the time \\
12 & Professional adults get it right half the time \\
18 & A talented adult with years of experience will get this half the time \\
\end{tabularx}
}

Now consider a regular person with no \term{expertise}. This person will roll d6 + d4.
They cannot overcome a challenge with 12,
and they will fail at relatively basic challenges of \acronymDL 5 half of the time.\\

\section*{Credits, References \& Bibliography}

Chapter header image credits: \cite{dice_image}

\printbibliography[heading=none]
