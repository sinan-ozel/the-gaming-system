\documentclass{LegrandOrangeTufteBook}

% Add some TODO notes for the writer's own use.
\usepackage[disable]{todonotes}

% Used for citations & references
\usepackage{biblatex}
\addbibresource{inc/citations.bib}

% Compile using UTF-8
\usepackage[utf8]{inputenc}

% This package combines longtable and tabularx
\usepackage{xltabular}

% Add the highlighted notes that appear when the reader hovers
\usepackage{acro}
\usepackage{pdfcomment}

% Get some nice pre-defined colors.
% Start here: https://ctan.math.ca/tex-archive/macros/latex/contrib/xkcdcolors/xkcdcolors-manual.pdf
% See the git page: https://github.com/Rmano/xkcdcolors
% See the full list here: https://xkcd.com/color/rgb/
% See here for the story: https://blog.xkcd.com/2010/05/03/color-survey-results/
\usepackage{xkcdcolors}

% Define coloring related to terms
% I am using "Contrasting Palette 1 & 2" from https://venngage.com/tools/accessible-color-palette-generator
\definecolor{ocre}{RGB}{196, 70, 1} % Define the color used for highlighting throughout the boo

\colorlet{colorBgMundane}{xkcdCloudyBlue}
\colorlet{colorBgModern}{xkcdPowderBlue}
\colorlet{colorBgMedieval}{xkcdSandy}
\colorlet{colorBgBeyondMundane}{xkcdSpearmint}

\colorlet{colorNatureDark}{xkcdDirtyOrange}
\colorlet{colorNature}{xkcdLightMustard}
\colorlet{colorNatureLight}{xkcdParchment}
\colorlet{colorNatureContour}{xkcdOffWhite}

\colorlet{colorCapabilityDark}{xkcdLightPlum}
\colorlet{colorCapability}{xkcdLightPink}
\colorlet{colorCapabilityLight}{xkcdVeryLightPink}
\colorlet{colorCapabilityContour}{xkcdVeryLightPink}

\colorlet{colorBonusDark}{xkcdMutedBlue}
\colorlet{colorBonus}{xkcdBabyBlue}
\colorlet{colorBonusLight}{xkcdIce}
\colorlet{colorBonusContour}{white}

\colorlet{colorPropertyDark}{xkcdDirtyOrange}
\colorlet{colorProperty}{xkcdLightMustard}
\colorlet{colorPropertyLight}{xkcdParchment}
\colorlet{colorPropertyContour}{xkcdOffWhite}

\colorlet{colorBeyond}{xkcdHunterGreen}

\colorlet{colorParchment}{orange!30!black!10}

% TODO: Remove colorCoreCompetency if not being used.
\colorlet{colorCoreCompetency}{colorCapabilityDark}
\newcommand{\termCore}[1]{\textcolor{colorCapabilityDark}{#1}}
\newcommand{\termBonus}[1]{\textcolor{colorBonusDark}{#1}}
\newcommand{\termDifficultyLevel}[1]{\textcolor{colorPropertyDark}{#1}}


% TODO: This needs to be black & boldface
\definecolor{colorTerm}{RGB}{196, 70, 1}
\newcommand{\term}[1]{\emph{#1}}
\newcommand{\termBeyond}[1]{\textcolor{colorBeyond}{#1}}

% TODO: Currently not using the following, remove if I never wind up using it.
\newcommand{\termClass}[1]{\textcolor{xkcdSpearmint}{#1}}

% Add the nice side tabs on the tables.
% Usage: Put into the rightmost cell in a tabularx environment.
% Example: \sideTab{LightBlue}{Modern}
\newcommand{\sideTab}[2]{\cellcolor{#1} \rotatebox[origin=l]{270}{#2}}
\newcommand{\rightSideTab}[2]{\cellcolor{#1} \rotatebox[origin=l]{270}{#2}}
\newcommand{\leftSideTab}[2]{\cellcolor{#1} \rotatebox[origin=r]{90}{#2}}


% Add color to table cells
% See here: https://tex.stackexchange.com/questions/50349/color-only-a-cell-of-a-table
\usepackage{colortbl}

% Use this package to break up a list into multiple columns
\usepackage{multicol}

% Add styling to emphasized paragraphs and character sheets
\usepackage{mdframed}

% TODO: Remove the following
% \global\mdfdefinestyle{d20ComparisonBoxStyle}{
% 	linewidth=1pt,
% 	linecolor=xkcdSandy,
% 	frametitlebackgroundcolor=xkcdPale,
% 	backgroundcolor=xkcdOffWhite,
% 	frametitlefont=\bfseries\sffamily\footnotesize,
% 	frametitlerule=true,
% }





% Style the bullets in itemized lists
% https://tex.stackexchange.com/questions/42805/what-are-original-itemize-bullet-definitions
\renewcommand\labelitemi{\textbullet}
\renewcommand\labelitemii{\normalfont\bfseries \textendash}
\renewcommand\labelitemiii{\textasteriskcentered}
\renewcommand\labelitemiv{\textperiodcentered}


\newcommand{\hover}[2]{
	\pdfmarkupcomment[markup=Highlight,disable=false,color=LightYellow]{#1}{#2}
}

\newcommand{\acronym}[2]{
	\pdfmarkupcomment[markup=Underline,disable=false,color=colorTerm]{#1}{#2}
}

\newcommand{\acronymDL}{
	\pdfmarkupcomment[markup=Underline,disable=false,color=colorPropertyDark]{DL}{
		Difficulty Level (DL): A roll has to beat this level number to succeed.
	}
}

\newcommand{\termMagic}[1]{
	\pdfmarkupcomment[markup=Underline,disable=false,color=colorPropertyDark]{#1}{
		Magic refers not only to fantasy literature magic, but also to everyday magical thinking, such as jinxes, astrology and Tarot and even surreal experiences such as inspiration drawn from works of art.
		There is no clear delineation from the ordinary use of the word ``magic'' or ``magical'' to supernatural effects in fantasy --- it is a scale.
	}
}

\newcommand{\footnoteMundaneAnimals}{Small mammals, large mammals, small reptiles, large reptiles, small birds, birds of prey}

\usepackage{wrapfig}

% Add tikzpicture to use for sheets
\usepackage{tikz}
\usetikzlibrary{positioning, calc, shadings, decorations.pathmorphing}

% Font for handwriting. Run udpmap to compile.
\usepackage{emerald}

% Font for titles. Run udpmap to compile.
\usepackage{comfortaa}

% Is this used? Test and remove if not used.
\usepackage{makecell}

% Useful for creating environments in an easier format.
\usepackage{environ}

% Expand table cells into multiple rows
\usepackage{multirow}


% Set and reset variables based on the hack here:
% https://tex.stackexchange.com/questions/37094/what-is-the-recommended-way-to-assign-a-value-to-a-variable-and-retrieve-it-for
\usepackage{pgfkeys}

% This is for the marginfigure environment.
\usepackage{sidenotes}

\newcommand{\setval}[1]{\pgfkeys{/variables/#1}}
\newcommand{\getval}[1]{\pgfkeysvalueof{/variables/#1}}
\newcommand{\declare}[1]{%
 \pgfkeys{
  /variables/#1.is family,
  /variables/#1.unknown/.style = {\pgfkeyscurrentpath/\pgfkeyscurrentname/.initial = ##1}
 }%
}

\declare{}

\newcommand{\cardwidth}{}
\newcommand{\sheetwidth}{}

\newenvironment{actorSheet}[5]{
	\setval{widthNature = 40mm}
	\setval{widthCapability = 25mm}
	\setval{widthBonus = 25mm}
	\setval{widthProperty = 25mm}
	\setval{widthExpertise = 25mm}

	\renewcommand{\cardwidth}{#2}
	\renewcommand{\sheetwidth}{#2}
	% TODO: Add height

	\begin{figure}[!h]

	\begin{tikzpicture}
		\draw[color=#5,line width=2mm]
			(#2, 0) coordinate (#1_SE) {} -- (0,  0) coordinate (#1_SW) {}
		;
		\draw[color=#5,line width=2pt]
			(0, #3) coordinate (#1_NW) {} -- (#2,#3) coordinate (#1_NE) {}
	   ;
	   \node[below right=1.5mm and -7.5pt of #1_NW,anchor=north west,inner sep=0,outer sep=0] (#1_title) {
			\setlength{\extrarowheight}{1.6mm}
			\setlength{\tabcolsep}{0mm}
			\setlength{\arraycolsep}{0mm}
			\begin{tabular}{@{}p{#2}}
				\cellcolor{#5} \hspace*{10mm} \ECFDecadenceInTheDarkCondensed #4\\[.9mm]
			\end{tabular}
	   };
}{
	\end{tikzpicture}

	\end{figure}
}
% TODO: Adjust the hspace depending on the card dimensions. Package calc or pgf?


\newenvironment{actorCardLetterSizeFitToPage}[3]{
	\begin{actorSheet}{#1}{\textwidth}{1.29411764706 * \textwidth}{#2}{#3}

	\setval{widthNature = .3\sheetwidth}
	\setval{widthCapability = 29.75mm}
	\setval{widthBonus = 29.75mm}
	\setval{widthProperty = 15mm}
	\setval{widthExpertise = 26.75mm}
}{
	\end{actorSheet}
}

\newenvironment{actorCardStandardSize}[3]{
	\begin{actorSheet}{#1}{2.5in}{3.5in}{#2}{#3}
}{
	\end{actorSheet}
}

\newenvironment{actorCardTarotSize}[3]{
	\begin{actorSheet}{#1}{7cm}{12cm}{#2}{#3}
}{
	\end{actorSheet}
}

\newenvironment{actorCardMiniEuro}[3]{
	\begin{actorSheet}{#1}{44mm}{68mm}{#2}{#3}
	\setval{widthNature = \sheetwidth}
	\setval{widthCapability = \sheetwidth}
	\setval{widthExpertise = 20mm}
}{
	\end{actorSheet}
}

\newenvironment{actorCardMiniAmerican}[3]{
	\begin{actorSheet}{#1}{41mm}{63mm}{#2}{#3}
}{
	\end{actorSheet}
}

\NewEnviron{natureBox}[1]{
	\node[
			draw=colorNatureDark,
			below left=11 mm and 0 in of #1_NW,
			top color=colorNature,
			middle color=colorNatureLight,
			bottom color=colorNatureContour,
			anchor=north west,
			inner ysep=0mm,
			inner xsep=0mm,
	] (#1_natureBox){
		\BODY
	};
}

\NewEnviron{natureTable}[1]{
	\begin{tabularx}{\getval{widthNature} - 3pt}[t]{Xr}
		\multicolumn{2}{c}{
			\cellcolor{colorNatureDark}
			\textcolor{colorNatureContour}{
				\comfortaa\footnotesize #1
			}
		} \\
		\BODY \\
	\end{tabularx}
}




\NewEnviron{capabilitiesBox}[1]{
	\node[
			draw=colorCapabilityDark,
			below left=4 mm and 0 mm of #1_natureBox.south west,
			top color=colorCapability,
			middle color=colorCapabilityLight,
			bottom color=colorCapabilityContour,
			anchor=north west,
			inner ysep=0mm,
			inner xsep=0mm,
	] (#1_capabilitiesBox){
		\BODY
	};
}

\NewEnviron{capabilitiesTable}[1]{
	\begin{tabularx}{\getval{widthCapability}}[t]{Xr}
		\multicolumn{2}{c}{
			\cellcolor{colorCapabilityDark}
			\textcolor{colorCapabilityContour}{
				\comfortaa\footnotesize #1
			}
		} \\
		\BODY \\
	\end{tabularx}
}

\NewEnviron{speciesBonusBox}[1]{
	\node[
			draw=colorBonusDark,
			below left=4 mm and 0 mm of #1_capabilitiesBox.south west,
			top color=colorBonus,
			middle color=colorBonusLight,
			bottom color=colorBonusContour,
			anchor=north west,
			inner ysep=0mm,
			inner xsep=0mm
	] (#1_speciesBonusBox){
		\BODY
	};
}

\NewEnviron{bonusTable}[1]{
	\begin{tabularx}{\getval{widthBonus}}[t]{Xr}
		\multicolumn{2}{c}{
			\cellcolor{colorBonusDark}
			\textcolor{colorBonusContour}{
				\comfortaa\footnotesize #1
			}
		} \\
		\BODY
	\end{tabularx}
}

\NewEnviron{expertiseBox}[1]{
	\node[
		draw=colorBonusDark,
		top color=colorBonus,
		middle color=colorBonusLight,
		bottom color=colorBonusContour,
		font=\normalsize,
		below left=4mm and 0mm of #1_capabilitiesBox.south west,
		anchor=north west,
		line width=.5mm,
		inner ysep=0mm,
		inner xsep=0mm
	] (#1_tableExpertise){
		\BODY
	};
}

\newcommand{\expertiseTable}[4]{
	\begin{tabular}[t]{c|c|c|c}
		\rowcolor{colorBonusDark}
		\sffamily\footnotesize\textcolor{colorBonusContour}{Training}	&
		\sffamily\footnotesize\textcolor{colorBonusContour}{Skills}		&
		\sffamily\footnotesize\textcolor{colorBonusContour}{Knowledge}	&
		\sffamily\footnotesize\textcolor{colorBonusContour}{Social} \\
		\begin{tabularx}{\getval{widthExpertise}}[t]{Xr}#1\end{tabularx}	&
		\begin{tabularx}{\getval{widthExpertise}}[t]{Xr}#2\end{tabularx} 	&
		\begin{tabularx}{\getval{widthExpertise}}[t]{Xr}#3\end{tabularx}	&
		\begin{tabularx}{\getval{widthExpertise}}[t]{Xr}#4\end{tabularx}
	\end{tabular}
}

\NewEnviron{physiologyBox}[1]{
	\node[
		draw=colorBonusDark,
		font=\normalsize,
		below left=4mm and 0mm of #1_tableExpertise,
		anchor=north west,
		line width=.5mm,
		inner ysep=0mm,
		inner xsep=0mm
	] (#1_tablePhysiology){
		\BODY
	};
}


\newcommand{\nature}[2]{
	\\[-3mm]
	\footnotesize #1 & \ECFAugie\footnotesize #2 \\
	\hline
}

\newcommand{\capability}[2]{
	\\[-3mm]
	\footnotesize #1 & \ECFAugie\footnotesize #2 \\
	\hline
}

\newcommand{\bonus}[2]{
	\footnotesize #1 & \footnotesize #2 \\
	\hline
}

\newcommand{\expertise}[2]{
	\ECFAugie\footnotesize #1 & \ECFAugie\footnotesize #2 \\
	\hline
}

% \newcommand{\propertyLine}[2]{
% 	\footnotesize #1 & \footnotesize #2 \\
% }


% \newcommand{\labeledFrame}[4]{
% 	\node[draw=xkcdLightPlum,top color=xkcdLightPink,middle color=xkcdVeryLightPink,bottom color=white,font=\normalsize,#3,anchor=north west,line width=.5mm,inner ysep=1mm,inner xsep=2mm] (#1){
% 		#4
% 	};
% 	\node[text=xkcdVeryLightPink,fill=xkcdLightPlum,node font=\sffamily,below right=0.2mm and -0.1mm of #1.north west,anchor=south west] (#1Label)  {
% 		\hspace*{5mm}#2
% 	};
% 	\coordinate[below right=.2mm and -.1mm of #1Label.north east,anchor=west] (#1A);
% 	\coordinate[right=-.1mm of #1Label.south east,anchor=west] (#1B);
% 	\coordinate[left=.6mm of #1.north east,anchor=south west] (#1C);
% 	\draw[xkcdLightPlum,fill=xkcdLightPlum] (#1A) -- (#1B) -- (#1C) -- (#1A);
% }


% Add a "parchment" environment for shaded paragraphs
% https://texample.net//tikz/examples/framed-tikz/

%%%%%%%%%%%%%%%%%%% BEGIN %%%%%%%%%%%%%%%%%%%%%%
\usepackage{framed}
\pgfmathsetseed{1} % To have predictable results
\pgfdeclarelayer{background}
\pgfsetlayers{background,main}

% define styles for the normal border and the torn border
\tikzset{
  normal border/.style={orange!30!black!10, decorate,
     decoration={random steps, segment length=2.5cm, amplitude=.7mm}},
  torn border/.style={orange!30!black!5, decorate,
     decoration={random steps, segment length=.5cm, amplitude=1.7mm}}}

% Macro to draw the shape behind the text, when it fits completly in the
% page
\def\parchmentframe#1{
\tikz{
  \node[inner sep=2em] (A) {#1};  % Draw the text of the node
  \begin{pgfonlayer}{background}  % Draw the shape behind
  \fill[normal border]
        (A.south east) -- (A.south west) --
        (A.north west) -- (A.north east) -- cycle;
  \end{pgfonlayer}}}

% Macro to draw the shape, when the text will continue in next page
\def\parchmentframetop#1{
\tikz{
  \node[inner sep=2em] (A) {#1};    % Draw the text of the node
  \begin{pgfonlayer}{background}
  \fill[normal border]              % Draw the ``complete shape'' behind
        (A.south east) -- (A.south west) --
        (A.north west) -- (A.north east) -- cycle;
  \fill[torn border]                % Add the torn lower border
        ($(A.south east)-(0,.2)$) -- ($(A.south west)-(0,.2)$) --
        ($(A.south west)+(0,.2)$) -- ($(A.south east)+(0,.2)$) -- cycle;
  \end{pgfonlayer}}}

% Macro to draw the shape, when the text continues from previous page
\def\parchmentframebottom#1{
\tikz{
  \node[inner sep=2em] (A) {#1};   % Draw the text of the node
  \begin{pgfonlayer}{background}
  \fill[normal border]             % Draw the ``complete shape'' behind
        (A.south east) -- (A.south west) --
        (A.north west) -- (A.north east) -- cycle;
  \fill[torn border]               % Add the torn upper border
        ($(A.north east)-(0,.2)$) -- ($(A.north west)-(0,.2)$) --
        ($(A.north west)+(0,.2)$) -- ($(A.north east)+(0,.2)$) -- cycle;
  \end{pgfonlayer}}}

% Macro to draw the shape, when both the text continues from previous page
% and it will continue in next page
\def\parchmentframemiddle#1{
\tikz{
  \node[inner sep=2em] (A) {#1};   % Draw the text of the node
  \begin{pgfonlayer}{background}
  \fill[normal border]             % Draw the ``complete shape'' behind
        (A.south east) -- (A.south west) --
        (A.north west) -- (A.north east) -- cycle;
  \fill[torn border]               % Add the torn lower border
        ($(A.south east)-(0,.2)$) -- ($(A.south west)-(0,.2)$) --
        ($(A.south west)+(0,.2)$) -- ($(A.south east)+(0,.2)$) -- cycle;
  \fill[torn border]               % Add the torn upper border
        ($(A.north east)-(0,.2)$) -- ($(A.north west)-(0,.2)$) --
        ($(A.north west)+(0,.2)$) -- ($(A.north east)+(0,.2)$) -- cycle;
  \end{pgfonlayer}}}

% Define the environment which puts the frame
% In this case, the environment also accepts an argument with an optional
% title (which defaults to ``Example'', which is typeset in a box overlaid
% on the top border
\newenvironment{parchment}[1][Example]{%
  \def\FrameCommand{\parchmentframe}%
  \def\FirstFrameCommand{\parchmentframetop}%
  \def\LastFrameCommand{\parchmentframebottom}%
  \def\MidFrameCommand{\parchmentframemiddle}%
  \vskip\baselineskip
  \MakeFramed {\FrameRestore}
  \noindent\tikz\node[inner sep=1ex, draw=black!20,fill=white,
          anchor=west, overlay] at (0em, 2em) {\sffamily#1};\par}%
{\endMakeFramed}
%%%%%%%%%%%%%%%%%%%% END %%%%%%%%%%%%%%%%%%%%%%%


% Side notes
\NewEnviron{formula}[1]{
	\begin{center}
		\begin{minipage}{3.5in}
			\begin{parchment}[#1]
				\centering
				\BODY
			\end{parchment}
		\end{minipage}
	\end{center}
}

\NewEnviron{marginNote}{
	\marginpar{
		\begin{mdframed}[backgroundcolor=colorParchment,linewidth=0pt]
			\footnotesize
			\BODY
		\end{mdframed}
	}
}

% Emphasis paragraph
\global\mdfdefinestyle{emphasisParagraphFrameStyle}{
	topline=false,
	bottomline=false,
	rightline=false,
	skipabove=\topsep,
	skipbelow=\topsep,
	linecolor=xkcdBottleGreen,
	linewidth=3pt,
}

\newenvironment{emphasisParagraph}{
	\begin{quote}
	\begin{mdframed}[style=emphasisParagraphFrameStyle]
	\em
}{
	\end{mdframed}
	\end{quote}
}




\newcommand{\listUniversalTraining}{
	\begin{itemize}[leftmargin=.5cm]
		\item Animal Handling (...)\footnote{Small mammals, large mammals, small reptiles, large reptiles, small birds, birds of prey}
		\item Artistic Painting
		\item Baking
		\item Brewing
		\item Carpentry
		\item Ceramics
		\item Cooking
		\item Drawing
		\item Embroidery
		\item Knitting
		\item Language (...)
		\item Lockpicking
		\item Massage
		\item Musical Instrument (...)
		\item Roof tiling
		\item Sleight of Hand
		\item Sailing
		\item Shoemaking
		\item Survival
		\item Swimming
		\item Winemaking
		\item Tailoring
  		\item Trap Making
	\end{itemize}
}

\newcommand{\listTalents}{
	\begin{itemize}[leftmargin=.5cm]
		\item Acting
		\item Assertion
		\item Deception
		\item Joking
		\item Listening
		\item Skepticism
		\item Taurascatics
	\end{itemize}
}

\newcommand{\listUniversalKnowledge}{
	\begin{itemize}[leftmargin=.5cm]
		\item Bookkeeping
		\item Algebra
		\item Astronomy
		\item Herbology
		\item Human Anatomy
		\item Philosophy
		\item Physics
		\item Politics
		\item Theology (...)
	\end{itemize}
}




\newcommand{\listModernTraining}{
	\begin{itemize}[leftmargin=.5cm]
		\item Bouldering
		\item Clinical Skills
		\item Cycling
		\item Dancing (...)\footnote{Ballroom, Capoeira, Hiphop, Latin, Tango}
		\item Defensive Martial Art (...)
		\item Driving (Cars)
		\item Firefighting
		\item First Aid
		\item Gymnastics
		\item Household Repair
		\item Interior Design
		\item Kickboxing
		\item Knife Fighting
		\item Medical Equipment (...)
		\item Mixology
		\item Offensive Martial Art (...)
		\item Photography
		\item Plumbing
		\item Riding (Motorcycle)
		\item Wrestling
	\end{itemize}
}

\newcommand{\listModernKnowledge}{
	\begin{itemize}[leftmargin=.5cm]
		\item Accounting
		\item Architecture
		\item Art History
		\item Astronomy
		\item Biology
		\item Calculus
		\item Cinema
		\item Civil Engineering
		\item Computer Art
		\item Computing Concepts
		\item Criminal Law
		\item Economics\footnote{Macro, unless otherwise specified.}
		\item Electrical Engineering
		\item Electronics
		\item Firearms
		\item Mechanics
		\item Music Theory
		\item Computer Networks
		\item Physics
		\item Psychology
		\item Sociology
		\item Theatre
		\item World History
		\item Writing
	\end{itemize}
}

\newcommand{\listModernTalents}{
	\begin{itemize}[leftmargin=.5cm]
		\item Fashion
	\end{itemize}
}



\newcommand{\listMedievalSkills}{
	\begin{itemize}[leftmargin=.5cm]
		\item Blacksmithing
		\item Locksmithing
		\item Riding (Horse)
		\item Riding (Camel)
		\item Wheelwrighting
		\item Tanning
	\end{itemize}
}

\newcommand{\listMedievalTraining}{
	\begin{itemize}[leftmargin=.5cm]
		\item Archery
		\item Maces
		\item Swords
		\item Nunchakus
		\item Off-hand Dagger Use
	\end{itemize}
}

\newcommand{\listMedievalKnowledge}{
}

\newcommand{\listFantasyTalents}{
	\begin{itemize}[leftmargin=.5cm]
		\item Clairvoyance
		\item Mediumship
		\item Telekinesis
		\item Pyrokinesis
	\end{itemize}
}

\newcommand{\listFantasyTraining}{
	\begin{itemize}[leftmargin=.5cm]
		\item Animal Handling (...)\footnote{Gigantic birds, gigantic arachnids, etc\ldots}
		\item Riding (Griffin)
		\item Riding (Dragon)
	\end{itemize}
}

\newcommand{\listFantasyKnowledge}{
	\begin{itemize}[leftmargin=.5cm]
		\item Alchemy
		\item Astrology
		\item Curses \& Hexes
		\item Demonology
		\item Necromancy
		\item Oneiromancy
		\item Summoning Rituals
		\item Tarot
		\item Thaumaturgy
	\end{itemize}
}

\begin{document}

\chapterimage{image/baby_faces.jpg}
\chapterspaceabove{2.75cm}
\chapterspacebelow{5.25cm}


\chapter*{Character Creation}

Generating a character like writing a social network profile or a CV: Based on the character idea, formulate their \term{core capabilities}.
Then based on their years of experience, list their \term{skills}, \term{talent} and \term{knowledge}.
List their \term{income} and \term{renown} based on their backstory.
Finally, based on their species and backstory, determine their \term{physiological bonuses}.
\marginpar{
	\footnotesize
	\term{Mundane} characters are humans in a modern setting. Spells, powers \& overpowered \term{core capabilities} take the character \term{beyond the mundane}.
}

\section*{Core Capabilities}

\term{Core capabilities} determine which dice to roll \& add to overcome challenges.
\marginpar{
	\footnotesize
	For d20 players:
	imagine \term{core capabilities} as dice that replace d20.
}

\coreCompetencyTable{2d6}{d4 + d6}{2d4}{d4 + d6}{2d4}{2d4}{4d6}{2d4}

To create a \term{mundane} character,
\marginpar{
	\footnotesize
	\begin{tabular}{rl}
d4 & Child \\
d6 & Teenager \\
2d4 & Not your strong suit \\
d4 + d6 & An average adult \\
2d6 & This is your strong suit \\
\end{tabular}
}
assign 2d4 if they are bad,
2d6 if they are good, and d4 + d6 if they are average.
d6 is for teenagers, literally or figuratively.\footnote{
 	Larger dice (d8, d10, ...) are reserved for heroes, superheroes and supernatural characters.
 	These are action movie protoganists, villains, comic book heroes, vampires.
 	The largest dice are for divine beings and legendary characters.
 	A \term{third dice} is for representing dangerous, forbidden magic and alien technology:
 	they increase the sum, but also the risk of rolling double ones, which means that a \term{misfortune} happened.
}

\section*{Expertise}

These are bonuses added to dice rolls. Assign these bonuses based on the character's background: Bonus equals the years of experience, and after the tenth year, count each decade as one point. Less than six months counts as half-year. Flesh out as much as desired, extend the \term{knowledge}, \term{skills} and \term{talents} depending on the setting and the focus of the campaign. The final bonuses need to make sense to \emph{all} other players when you are introducing your character.\\

\marginpar{
	\footnotesize
	\begin{tabular}{rl}
Years & Bonus \\
\hline
1 yr & +1 \\
2 yrs & +2 \\
3 yrs & +3 \\
\multicolumn{2}{c}{...} \\
9 yrs & +9 \\
10 yrs & +10 \\
11 yrs & +10 \\
12 yrs & +10 \\
\multicolumn{2}{c}{...} \\
19 yrs & +10 \\
20 yrs & +11 \\
21 yrs & +11 \\
\multicolumn{2}{c}{...} \\
30 yrs & +12 \\
\end{tabular}
}


\newcommand{\explainEE}{The character took 4 years of classes related to Electrical Engineering.This gives +4 bonus under \term{knowledge}.  His work between the ages 24 and 26 involved components used in electrical engineering. This sums up to 6 years, or a bonus of +6.}
\newcommand{\explainEcon}{The character took multiple classes during the master's degree. I decided that this sums up 2 or 3, and went with the lower number.}
\newcommand{\explainFinance}{Two years of classes on finance and no work experience.}
\newcommand{\explainCycling}{The character can cycle: they taught themselves as a kid. For about six summers, they spent a lot of summers cycling. However, it's just the summer, and not the full year. I divide 6 years in half, and get +3 as the bonus.}
\newcommand{\explainDriving}{Driving is something that you would take lessons to learn. Furthermore, it is a mixture of theoretical knowledge and training. So it is classified as a \term{skill}. The character has been driving regularly since the age of 22, so the bonus is +5.}
\newcommand{\explainProgramming}{He started programming at the age of 9, did that as a hobby pretty much every year until the age of 22. I decide to divide this by half - it's not equivalent of full time work. However, between the ages 22 and 26, he worked full time programming a system, so I count that as 4. The total is 10.}
\newcommand{\explainSkepticism}{When the character started work, they found themselves in situations where they need to understand when people are being deceptive. This is the only social talent that this character has. While difficult to quantify in terms of years of experience, the player decides that this is fair based on the character history \& concept.}


\begin{center}
	\begin{tabular}{lr|lr|lr|}
		\multicolumn{2}{c}{Knowledge}							&	\multicolumn{2}{c}{Skills}				&	\multicolumn{2}{c}{Talents} \\
		\hline
		Electrical Engineering & \hover{+6}{\explainEE}			& 	Cycling & \hover{+3}{\explainCycling}		& 	Skepticism &  \hover{+4}{\explainSkepticism} \\
		Computing Concepts & \hover{+10}{\explainProgramming}	& 	Dancing (Latin) & +3				& 	 &  \\
		Economics & \hover{+2}{\explainEcon}					& 	Driving (Car) & \hover{+5}{\explainDriving}	& 	& \\
	\end{tabular}
\end{center}



% \footnote{
% 	If worked on something more than half of a year, that's a full year, round up. If less than a year (for instance, summers only) divide the total number of years by half.
% }


\term{Knowledge} is pure theory. In the modern setting, these are 100-level university courses. They can be purely academic, or they can be useful at high-income jobs.\\
\term{Skills} are everything that you learn mainly by doing:
they will have a technical component, and may require taking some classes, but to get beyond the basic, you need to keep using skills.
In the modern setting, think of everything you would learn in night classes, or by working with an individual, or with a coach or a trainer.
In other settings, they may be used through apprenticeship.
These may be useful in a medium-income jobs.\\
\term{Talents} cannot be easily taught, you have to learn these by doing.
In the modern setting, these are useful in social situations.
However, in other settings, they can include any other talent that is innate or developed.
They are transferrable across jobs and even different campaign settings.\\

If something can be taught with less then one hour of instructions, that is not an area of \term{expertise}.
This is something that an adult should be able to do very easily after hearing the instructions,
so the \term{difficulty level} of using such an equipment will likely be 3 on simple checks,
and should be determined only by the \term{core capability}.
This includes crossbox and modern firearms.
\footnote{"Firearms", listed under knowledge, is about encyclopedic knowledge of firearms and their maintenance, not for shooting.}


% \section*{Resources: Passive Income \& Renown}

% House
% Social Support
% Work Network
% Criminal Network?


\section*{Character in Gameplay}

\subsection*{Character Example}

\marginpar{
	\vspace{7cm}
	\footnotesize
	Any income to the chraracter that does not involve a job is \term{passive income}. This can be rent from a real estate, parental support, or from investments. (Exact score determination needs to be decided, this can be a dice or a bonus.)\\

	\term{Renown} is the based on the number of you people who know of the character, without the character knowing them.
	To determine the score, get the number of people who would know of the chracter, and count the number of digits.
	For example, a researcher well known in their field could be 2 or even 3.
	An modern-era influencer could go up to 4 (1000 followers or more), 5, (10000 followers or more).
	A celebrity could be at 6 or 7, or even 8 if they are very famous.
	Most people should have 0 or, perhaps 1 if they have an online presence.
}


\begin{minipage}{\columnwidth}
\begin{center}
	\coreCompetencyTable{2d6}{d4 + d6}{2d4}{d4 + d6}{2d4}{2d4}{4d6}{2d4}
	\resizebox{\columnwidth}{!}{
		\begin{tabular}{lr|lr|lr|}

			\multicolumn{6}{c}{} \\
			\multicolumn{2}{c}{Knowledge}	&	\multicolumn{2}{c}{Skills}		&	\multicolumn{2}{c}{Talents} \\
			\hline
			Electrical Engineering & +6		& 	Cycling & +3					& 	Skepticism & +4 \\
			Economics & +2					& 	Dancing (Latin) & +3			& 	& \\
			Marketing & +1					& 	Swimming & +2 					& 	& \\
			Finance & +2 					& 	Driving (Car) & +5				& 	& \\
			Accounting  & +1 				& 	 &  							& 	& \\
			Philosophy  & +1 				& 	 &  							& 	& \\
			Physics  & +1 					& 	 &  							& 	& \\
			Chemistry  & +1  				& 	 &  							& 	& \\
			Computing Concepts & +10  		& 	 &  							& 	& \\
			\multicolumn{6}{c}{} \\
 			\multicolumn{2}{c}{Resources}	&	\multicolumn{2}{c}{Physiology}	&	\multicolumn{2}{c}{Items} \\
			\hline
			Passive Income & +1				& Sight	& +6						&  Mobile Phone & \\
			Renown & 						& Hearing &	+2						&  & \\
			 & 								& Strength, Upper Body & +1			&  & \\
			 & 								& Strength, Core & +1				&  & \\
		\end{tabular}
	}
\end{center}
\end{minipage}
\todo{Need to add the rules for physilogical bonuses.}
\todo{Add size to physiology.}


\subsection*{Dice Roll Examples}

This character is a technically proficient young professional. The following are a list of situations and the rolls that the character above would need to make.
Some of these situations are in the modern setting, some of them are in fantasy settings. The same character can be used without modification in all of them.

\subsubsection*{Mundane}

\begin{center}
	\begin{xltabular}{\textwidth}{Xll}
				Situation												& Challenge 								& Roll \\
\rowcolor{xkcdEggShell}	They need to find a solution to an engineering problem at work.				& \termCore{Reasoning} + Electrical Engineering		& 2d6 + 6 \\
				They want to meet someone at a party.								& \termCore{Social Awareness}					& 2d4 \\
\rowcolor{xkcdEggShell}	They are in an emergency situation where a fire started, and they need to react	& \termCore{Situational Awareness} + Firefighting 		& d4 + d6 + 0 \\
		 		See through deception during conversation							& \termCore{Reasoning} + Skepticism				& d4 + d6 + 5 \\
\rowcolor{xkcdEggShell}	Deceive someone during conversation 								& \termCore{Communication} + Deception			& d4 + d6 + 0 \\
		 		Throw a fist at an unsuspecting individual							& \termCore{Coordination} + Offensive Martial Art (Boxing)	& 2d4 + 0 \\
	\end{xltabular}
\end{center}

\subsubsection*{Beyond the Mundane}

\begin{center}
	\begin{xltabular}{\textwidth}{Xll}
				Situation												& Challenge 								& Roll \\
\rowcolor{xkcdEggShell}	They need to identify some potions through sight, smell \& taste. 				& \termCore{Situational Awareness} + Herbology		& d4 + d6 + 0 \\
	\end{xltabular}
\end{center}

\pagebreak
\section*{List of Expertise Areas}

\begin{center}
	\begin{xltabular}{\textwidth}{XXXl}
		Knowledge 								& Skills								&  Talents & \\
		\hline
		\footnotesize \listUniversalKnowledge	&\footnotesize \listUniversalTraining	& \footnotesize \listTalents 	 	& \sideTab{xkcdCloudyBlue}{All Settings} \\
		\footnotesize \listModernKnowledge		&\footnotesize \listModernTraining		& \footnotesize \listModernTalents	& \sideTab{xkcdPowderBlue}{Modern Setting} \\
		\footnotesize \listMedievalKnowledge	&\footnotesize \listMedievalTraining	& \footnotesize  				& \sideTab{xkcdSandy}{Medieval Setting} \\
		\footnotesize \listFantasyKnowledge		&\footnotesize \listFantasyTraining		& \footnotesize  \listFantasyTalents 	& \sideTab{xkcdSpearmint}{Beyond the Mundane} \\
	\end{xltabular}
\end{center}




\section*{Creating Characters for Different Campaigns}

\subsection*{Mundane}
\begin{center}
	\begin{xltabular}{\textwidth}{XX}
		Campaign & Character Creation \\
		\hline
\rowcolor{xkcdEggShell}    A group of high school friends come back to their home town when one of them passes away. As they mourn their friends' passing, they begin to suspect foulplay and a coverup... & Create characters as usual. \\
				A band of volunteer soldiers find themselves in a war. & Create characters as usual. \\
\rowcolor{xkcdEggShell}    In a longer campaign, a group of firegihters respond to crises in a particularly flammable \& dangerous city. & Create characters as usual. Obviously, the background should reflect their years in the firefighting force. \\
	\end{xltabular}
\end{center}


\subsection*{Beyond the Mundane}
\begin{center}
	\begin{xltabular}{\textwidth}{XX}
		Campaign & Character Creation \\
		\hline
\rowcolor{xkcdEggShell}    A band of talented teenagers investigate paranormal phenomenon. & Create teenagers with d6 in all \term{core capabilities}, except for their one special talent, which is 2d8.\\
   				Young group of friends are attending a secret school of magic. & Create kids with d4 in all \term{core capabilities}. However, also add in the fantasy knowledge and talents. Add a spellbook. \\

	\end{xltabular}
\end{center}

\section*{Balance}

\subsection*{Balancing Gameplay}

With this level of flexibility, finding the right difficulty for the challenges may be difficult. Look at the players' rolls, and find the maximum roll possible. For example, for an young professional,
this number is likely going to be 18: This is going to be a roll with a \term{core capability} where they are particulary talented, so this is going to be 2d6. The highest possible roll is 12. They probably have at least five or six years of experience in their job, so this brings us up to 18. This is the character's \term{mundane level}.

Plan a scene around one climatic challenge at this \term{difficulty level}, potentially for another \term{core capability} and another area of \term{expertise}. They should have a very low or zero probability of passing. However, using item bonuses, their \term{resources}, and planning, they should be able to bring this to a high probability event. There will still be the probability of a double-one, a \term{misfortune}. Even if they are very well prepared when they face the challenge, if they roll double ones, they may still pass, but a misfortune falls upon them.


\end{document}

