% Add some TODO notes for the writer's own use.
\usepackage[disable]{todonotes}

% This package combines longtable and tabularx
\usepackage{xltabular}

% Add the highlighted notes that appear when the reader hovers
\usepackage{acro}
\usepackage{pdfcomment}

% Get some nice pre-defined colors.
% Start here: https://ctan.math.ca/tex-archive/macros/latex/contrib/xkcdcolors/xkcdcolors-manual.pdf
% See the git page: https://github.com/Rmano/xkcdcolors
% See the full list here: https://xkcd.com/color/rgb/
% See here for the story: https://blog.xkcd.com/2010/05/03/color-survey-results/
\usepackage{xkcdcolors}

% Define coloring related to terms
% I am using "Contrasting Palette 1 & 2" from https://venngage.com/tools/accessible-color-palette-generator
\definecolor{ocre}{RGB}{196, 70, 1} % Define the color used for highlighting throughout the boo

\definecolor{colorCoreCompetency}{RGB}{91, 163, 0}
\newcommand{\termCore}[1]{\textcolor{colorCoreCompetency}{#1}}

\definecolor{colorTerm}{RGB}{196, 70, 1}
\newcommand{\term}[1]{\textcolor{colorTerm}{#1}}

\newcommand{\termClass}[1]{\textcolor{xkcdSpearmint}{#1}}

% Add the nice side tabs on the tables.
% Usage: Put into the rightmost cell in a tabularx environment.
% Example: \sideTab{LightBlue}{Modern}
\newcommand{\sideTab}[2]{\cellcolor{#1} \rotatebox[origin=l]{270}{#2}}

% Add color to table cells
% See here: https://tex.stackexchange.com/questions/50349/color-only-a-cell-of-a-table
\usepackage{colortbl}

% Use this package to break up a list into multiple columns
\usepackage{multicol}

% Add styling to emphasized paragraphs and character sheets
\usepackage{mdframed}

\global\mdfdefinestyle{emphasisParagraphFrameStyle}{
	topline=false,
	bottomline=false,
	rightline=false,
	skipabove=\topsep,
	skipbelow=\topsep,
	linecolor=xkcdBottleGreen,
	linewidth=3pt,
	leftmargin=0pt,
	rightmargin=0pt,
	innertopmargin=0pt,
	innerbottommargin=0pt
}

% Style the bullets in itemized lists
% https://tex.stackexchange.com/questions/42805/what-are-original-itemize-bullet-definitions
\renewcommand\labelitemi{\textbullet}
\renewcommand\labelitemii{\normalfont\bfseries \textendash}
\renewcommand\labelitemiii{\textasteriskcentered}
\renewcommand\labelitemiv{\textperiodcentered}


% Make it easy to create the core capabilities of a character sheet
\newcommand{\coreCompetencyTable}[8]{
	\begin{center}
	\resizebox{\columnwidth}{!}{
		\begin{tabular}{lr|clr|}
			\multicolumn{2}{c}{Mental}	&&	\multicolumn{2}{c}{Physical} \\
			\cline{1-2} \cline{4-5}
			Reasoning & #1				&& 	Coordination & #3 \\
			Situational Awareness & #2	&& 	Constitution & #4 \\
			\multicolumn{4}{c}{} \\
			\multicolumn{2}{c}{Social}	&&	\multicolumn{2}{c}{Innate} \\
			\cline{1-2} \cline{4-5}
			Communication & #5 			&&	Focus & #7 \\
			Social Awareness & #6 		&&	Creativity & #8 	\\
		\end{tabular}
	}
	\end{center}
}

\newcommand{\hover}[2]{
	\pdfmarkupcomment[markup=Highlight,disable=false,color=LightYellow]{#1}{#2}
}

\newcommand{\acronym}[2]{
	\pdfmarkupcomment[markup=Underline,disable=false,color=colorTerm]{#1}{#2}
}

\newcommand{\acronymDL}[1]{
	\pdfmarkupcomment[markup=Underline,disable=false,color=colorTerm]{DL}{
		Difficulty Level: A roll has to beat this level number to succeed.
	}
}

\newenvironment{characterSheet}{
	\begin{minipage}{\columnwidth}
	\begin{center}
}
{
	\end{center}
	\end{minipage}
}

\newcommand{\expertiseTable}[3]{
	\begin{tabular}{ccc}
		\multicolumn{3}{c}{} \\
		Knowledge & Skills & Talents \\
		\hline
		\begin{tabular}[t]{lr|}
			#1
		\end{tabular} &
		\begin{tabular}[t]{lr|}
			#2
		\end{tabular} &
		\begin{tabular}[t]{lr}
			#3
		\end{tabular} \\
	\end{tabular}
}

\newcommand{\physiologyTable}[3]{
	\begin{tabular}{ccc}
		\multicolumn{3}{c}{} \\
		Senses & Strength & Movement \\
		\hline
		\begin{tabular}[t]{lr|}
			#1
		\end{tabular} &
		\begin{tabular}[t]{lr|}
			#2
		\end{tabular} &
		\begin{tabular}[t]{lr}
			#3
		\end{tabular} \\
	\end{tabular}
}
% TODO: How to do speed and size?

\newenvironment{emphasisParagraph}{
	\begin{quote}
	\begin{mdframed}[style=emphasisParagraphFrameStyle]
	\em
}{
	\end{mdframed}
	\end{quote}
}