% Add some TODO notes for the writer's own use.
\usepackage[disable]{todonotes}

% Used for citations & references
\usepackage{biblatex}
\addbibresource{inc/citations.bib}

% Compile using UTF-8
\usepackage[utf8]{inputenc}

% This package combines longtable and tabularx
\usepackage{xltabular}

% Add the highlighted notes that appear when the reader hovers
\usepackage{acro}
\usepackage{pdfcomment}

% Get some nice pre-defined colors.
% Start here: https://ctan.math.ca/tex-archive/macros/latex/contrib/xkcdcolors/xkcdcolors-manual.pdf
% See the git page: https://github.com/Rmano/xkcdcolors
% See the full list here: https://xkcd.com/color/rgb/
% See here for the story: https://blog.xkcd.com/2010/05/03/color-survey-results/
\usepackage{xkcdcolors}

% Define coloring related to terms
% I am using "Contrasting Palette 1 & 2" from https://venngage.com/tools/accessible-color-palette-generator
\definecolor{ocre}{RGB}{196, 70, 1} % Define the color used for highlighting throughout the boo

\colorlet{colorBgMundane}{xkcdCloudyBlue}
\colorlet{colorBgModern}{xkcdPowderBlue}
\colorlet{colorBgMedieval}{xkcdSandy}
\colorlet{colorBgBeyondMundane}{xkcdSpearmint}

\colorlet{colorNatureDark}{xkcdDirtyOrange}
\colorlet{colorNature}{xkcdLightMustard}
\colorlet{colorNatureLight}{xkcdParchment}
\colorlet{colorNatureContour}{xkcdOffWhite}

\colorlet{colorCapabilityDark}{xkcdLightPlum}
\colorlet{colorCapability}{xkcdLightPink}
\colorlet{colorCapabilityLight}{xkcdVeryLightPink}
\colorlet{colorCapabilityContour}{xkcdVeryLightPink}

\colorlet{colorBonusDark}{xkcdMutedBlue}
\colorlet{colorBonus}{xkcdBabyBlue}
\colorlet{colorBonusLight}{xkcdIce}
\colorlet{colorBonusContour}{white}

\colorlet{colorPropertyDark}{xkcdDirtyOrange}
\colorlet{colorProperty}{xkcdLightMustard}
\colorlet{colorPropertyLight}{xkcdParchment}
\colorlet{colorPropertyContour}{xkcdOffWhite}

\colorlet{colorBeyond}{xkcdHunterGreen}

\colorlet{colorParchment}{orange!30!black!10}

% TODO: Remove colorCoreCompetency if not being used.
\colorlet{colorCoreCompetency}{colorCapabilityDark}
\newcommand{\termCore}[1]{\textcolor{colorCapabilityDark}{#1}}
\newcommand{\termBonus}[1]{\textcolor{colorBonusDark}{#1}}
\newcommand{\termDifficultyLevel}[1]{\textcolor{colorPropertyDark}{#1}}


% TODO: This needs to be black & boldface
\definecolor{colorTerm}{RGB}{196, 70, 1}
\newcommand{\term}[1]{\emph{#1}}
\newcommand{\termBeyond}[1]{\textcolor{colorBeyond}{#1}}

% TODO: Currently not using the following, remove if I never wind up using it.
\newcommand{\termClass}[1]{\textcolor{xkcdSpearmint}{#1}}

% Add the nice side tabs on the tables.
% Usage: Put into the rightmost cell in a tabularx environment.
% Example: \sideTab{LightBlue}{Modern}
\newcommand{\sideTab}[2]{\cellcolor{#1} \rotatebox[origin=l]{270}{#2}}
\newcommand{\rightSideTab}[2]{\cellcolor{#1} \rotatebox[origin=l]{270}{#2}}
\newcommand{\leftSideTab}[2]{\cellcolor{#1} \rotatebox[origin=r]{90}{#2}}


% Add color to table cells
% See here: https://tex.stackexchange.com/questions/50349/color-only-a-cell-of-a-table
\usepackage{colortbl}

% Use this package to break up a list into multiple columns
\usepackage{multicol}

% Add styling to emphasized paragraphs and character sheets
\usepackage{mdframed}

% TODO: Remove the following
% \global\mdfdefinestyle{d20ComparisonBoxStyle}{
% 	linewidth=1pt,
% 	linecolor=xkcdSandy,
% 	frametitlebackgroundcolor=xkcdPale,
% 	backgroundcolor=xkcdOffWhite,
% 	frametitlefont=\bfseries\sffamily\footnotesize,
% 	frametitlerule=true,
% }





% Style the bullets in itemized lists
% https://tex.stackexchange.com/questions/42805/what-are-original-itemize-bullet-definitions
\renewcommand\labelitemi{\textbullet}
\renewcommand\labelitemii{\normalfont\bfseries \textendash}
\renewcommand\labelitemiii{\textasteriskcentered}
\renewcommand\labelitemiv{\textperiodcentered}


\newcommand{\hover}[2]{
	\pdfmarkupcomment[markup=Highlight,disable=false,color=LightYellow]{#1}{#2}
}

\newcommand{\acronym}[2]{
	\pdfmarkupcomment[markup=Underline,disable=false,color=colorTerm]{#1}{#2}
}

\newcommand{\acronymDL}{
	\pdfmarkupcomment[markup=Underline,disable=false,color=colorPropertyDark]{DL}{
		Difficulty Level (DL): A roll has to beat this level number to succeed.
	}
}

\newcommand{\termMagic}[1]{
	\pdfmarkupcomment[markup=Underline,disable=false,color=colorPropertyDark]{#1}{
		Magic refers not only to fantasy literature magic, but also to everyday magical thinking, such as jinxes, astrology and Tarot and even surreal experiences such as inspiration drawn from works of art.
		There is no clear delineation from the ordinary use of the word ``magic'' or ``magical'' to supernatural effects in fantasy --- it is a scale.
	}
}

\newcommand{\footnoteMundaneAnimals}{Small mammals, large mammals, small reptiles, large reptiles, small birds, birds of prey}

\usepackage{wrapfig}

% Add tikzpicture to use for sheets
\usepackage{tikz}
\usetikzlibrary{positioning, calc, shadings, decorations.pathmorphing}

% Font for handwriting. Run udpmap to compile.
\usepackage{emerald}

% Font for titles. Run udpmap to compile.
\usepackage{comfortaa}

% Is this used? Test and remove if not used.
\usepackage{makecell}

% Useful for creating environments in an easier format.
\usepackage{environ}

% Expand table cells into multiple rows
\usepackage{multirow}


% Set and reset variables based on the hack here:
% https://tex.stackexchange.com/questions/37094/what-is-the-recommended-way-to-assign-a-value-to-a-variable-and-retrieve-it-for
\usepackage{pgfkeys}

% This is for the marginfigure environment.
\usepackage{sidenotes}

\newcommand{\setval}[1]{\pgfkeys{/variables/#1}}
\newcommand{\getval}[1]{\pgfkeysvalueof{/variables/#1}}
\newcommand{\declare}[1]{%
 \pgfkeys{
  /variables/#1.is family,
  /variables/#1.unknown/.style = {\pgfkeyscurrentpath/\pgfkeyscurrentname/.initial = ##1}
 }%
}

\declare{}

\newcommand{\cardwidth}{}
\newcommand{\sheetwidth}{}

\newenvironment{actorSheet}[5]{
	\setval{widthNature = 40mm}
	\setval{widthCapability = 25mm}
	\setval{widthBonus = 25mm}
	\setval{widthProperty = 25mm}
	\setval{widthExpertise = 25mm}

	% Not tested.
	\renewcommand{\cardwidth}{#2}
	\renewcommand{\sheetwidth}{#2}
	% TODO: Add height if this works

	\begin{figure}[!h]

	\begin{tikzpicture}
		\draw[color=#5,line width=2mm]
			(#2, 0) coordinate (#1_SE) {} -- (0,  0) coordinate (#1_SW) {}
		;
		\draw[color=#5,line width=2pt]
			(0, #3) coordinate (#1_NW) {} -- (#2,#3) coordinate (#1_NE) {}
	   ;
	   \node[below right=1.5mm and -7.5pt of #1_NW,anchor=north west,inner sep=0,outer sep=0] (#1_title) {
			\setlength{\extrarowheight}{1.6mm}
			\setlength{\tabcolsep}{0mm}
			\setlength{\arraycolsep}{0mm}
			\begin{tabular}{@{}p{#2}}
				\cellcolor{#5} \hspace*{10mm} \ECFDecadenceInTheDarkCondensed #4\\[.9mm]
			\end{tabular}
	   };
}{
	\end{tikzpicture}

	\end{figure}
}
% TODO: Adjust the hspace depending on the card dimensions. Package calc or pgf?


\newenvironment{actorCardLetterSizeFitToPage}[3]{
	\begin{actorSheet}{#1}{\textwidth}{1.29411764706 * \textwidth}{#2}{#3}
	\setval{widthNature = 40mm}
	\setval{widthCapability = 29.75mm}
	\setval{widthBonus = 29.75mm}
	\setval{widthProperty = 15mm}
	\setval{widthExpertise = 26.75mm}
}{
	\end{actorSheet}
}

\newenvironment{actorCardStandardSize}[3]{
	\begin{actorSheet}{#1}{2.5in}{3.5in}{#2}{#3}
}{
	\end{actorSheet}
}

\newenvironment{actorCardTarotSize}[3]{
	\begin{actorSheet}{#1}{7cm}{12cm}{#2}{#3}
}{
	\end{actorSheet}
}

\newenvironment{actorCardMiniEuro}[3]{
	\begin{actorSheet}{#1}{44mm}{68mm}{#2}{#3}
	\setval{widthCapability = 32.5mm}
	\setval{widthExpertise = 20mm}
}{
	\end{actorSheet}
}

\newenvironment{actorCardMiniAmerican}[3]{
	\begin{actorSheet}{#1}{41mm}{63mm}{#2}{#3}
}{
	\end{actorSheet}
}

\NewEnviron{natureBox}[1]{
	\node[
			draw=colorNatureDark,
			below left=11 mm and 0 in of #1_NW,
			top color=colorNature,
			middle color=colorNatureLight,
			bottom color=colorNatureContour,
			anchor=north west,
			inner ysep=0mm,
			inner xsep=0mm,
	] (#1_natureBox){
		\BODY
	};
}

\NewEnviron{natureTable}[1]{
	\begin{tabularx}{\getval{widthNature}}[t]{Xr}
		\multicolumn{2}{c}{
			\cellcolor{colorNatureDark}
			\textcolor{colorNatureContour}{
				\comfortaa\footnotesize #1
			}
		} \\
		\BODY \\
	\end{tabularx}
}




\NewEnviron{capabilitiesBox}[1]{
	\node[
			draw=colorCapabilityDark,
			below left=11 mm and 0 in of #1_natureBox,
			top color=colorCapability,
			middle color=colorCapabilityLight,
			bottom color=colorCapabilityContour,
			anchor=north west,
			inner ysep=0mm,
			inner xsep=0mm,
	] (#1_capabilitiesBox){
		\BODY
	};
}

\NewEnviron{capabilitiesTable}[1]{
	\begin{tabularx}{\getval{widthCapability}}[t]{Xr}
		\multicolumn{2}{c}{
			\cellcolor{colorCapabilityDark}
			\textcolor{colorCapabilityContour}{
				\comfortaa\footnotesize #1
			}
		} \\
		\BODY \\
	\end{tabularx}
}

\NewEnviron{speciesBonusBox}[1]{
	\node[
			draw=colorBonusDark,
			below left=4mm and 0mm of #1_capabilitiesBox,
			top color=colorBonus,
			middle color=colorBonusLight,
			bottom color=colorBonusContour,
			anchor=north west,
			inner ysep=0mm,
			inner xsep=0mm
	] (#1_speciesBonusBox){
		\BODY
	};
}

\NewEnviron{bonusTable}[1]{
	\begin{tabularx}{\getval{widthBonus}}[t]{Xr}
		\multicolumn{2}{c}{
			\cellcolor{colorBonusDark}
			\textcolor{colorBonusContour}{
				\comfortaa\footnotesize #1
			}
		} \\
		\BODY
	\end{tabularx}
}

\NewEnviron{expertiseBox}[1]{
	\node[
		draw=colorBonusDark,
		top color=colorBonus,
		middle color=colorBonusLight,
		bottom color=colorBonusContour,
		font=\normalsize,
		below left=4mm and 0mm of #1_capabilitiesBox,
		anchor=north west,
		line width=.5mm,
		inner ysep=0mm,
		inner xsep=0mm
	] (#1_tableExpertise){
		\BODY
	};
}

\newcommand{\expertiseTable}[4]{
	\begin{tabular}[t]{c|c|c|c}
		\rowcolor{colorBonusDark}
		\sffamily\footnotesize\textcolor{colorBonusContour}{Training}	&
		\sffamily\footnotesize\textcolor{colorBonusContour}{Skills}		&
		\sffamily\footnotesize\textcolor{colorBonusContour}{Knowledge}	&
		\sffamily\footnotesize\textcolor{colorBonusContour}{Social} \\
		\begin{tabularx}{\getval{widthExpertise}}[t]{Xr}#1\end{tabularx}	&
		\begin{tabularx}{\getval{widthExpertise}}[t]{Xr}#2\end{tabularx} 	&
		\begin{tabularx}{\getval{widthExpertise}}[t]{Xr}#3\end{tabularx}	&
		\begin{tabularx}{\getval{widthExpertise}}[t]{Xr}#4\end{tabularx}
	\end{tabular}
}

\NewEnviron{physiologyBox}[1]{
	\node[
		draw=colorBonusDark,
		font=\normalsize,
		below left=4mm and 0mm of #1_tableExpertise,
		anchor=north west,
		line width=.5mm,
		inner ysep=0mm,
		inner xsep=0mm
	] (#1_tablePhysiology){
		\BODY
	};
}


\newcommand{\nature}[2]{
	\\[-3mm]
	\footnotesize #1 & \ECFAugie\footnotesize #2 \\
	\hline
}

\newcommand{\capability}[2]{
	\\[-3mm]
	\footnotesize #1 & \ECFAugie\footnotesize #2 \\
	\hline
}

\newcommand{\bonus}[2]{
	\footnotesize #1 & \footnotesize #2 \\
	\hline
}

\newcommand{\expertise}[2]{
	\ECFAugie\footnotesize #1 & \ECFAugie\footnotesize #2 \\
	\hline
}

% \newcommand{\propertyLine}[2]{
% 	\footnotesize #1 & \footnotesize #2 \\
% }


% \newcommand{\labeledFrame}[4]{
% 	\node[draw=xkcdLightPlum,top color=xkcdLightPink,middle color=xkcdVeryLightPink,bottom color=white,font=\normalsize,#3,anchor=north west,line width=.5mm,inner ysep=1mm,inner xsep=2mm] (#1){
% 		#4
% 	};
% 	\node[text=xkcdVeryLightPink,fill=xkcdLightPlum,node font=\sffamily,below right=0.2mm and -0.1mm of #1.north west,anchor=south west] (#1Label)  {
% 		\hspace*{5mm}#2
% 	};
% 	\coordinate[below right=.2mm and -.1mm of #1Label.north east,anchor=west] (#1A);
% 	\coordinate[right=-.1mm of #1Label.south east,anchor=west] (#1B);
% 	\coordinate[left=.6mm of #1.north east,anchor=south west] (#1C);
% 	\draw[xkcdLightPlum,fill=xkcdLightPlum] (#1A) -- (#1B) -- (#1C) -- (#1A);
% }


% Add a "parchment" environment for shaded paragraphs
% https://texample.net//tikz/examples/framed-tikz/

%%%%%%%%%%%%%%%%%%% BEGIN %%%%%%%%%%%%%%%%%%%%%%
\usepackage{framed}
\pgfmathsetseed{1} % To have predictable results
\pgfdeclarelayer{background}
\pgfsetlayers{background,main}

% define styles for the normal border and the torn border
\tikzset{
  normal border/.style={orange!30!black!10, decorate,
     decoration={random steps, segment length=2.5cm, amplitude=.7mm}},
  torn border/.style={orange!30!black!5, decorate,
     decoration={random steps, segment length=.5cm, amplitude=1.7mm}}}

% Macro to draw the shape behind the text, when it fits completly in the
% page
\def\parchmentframe#1{
\tikz{
  \node[inner sep=2em] (A) {#1};  % Draw the text of the node
  \begin{pgfonlayer}{background}  % Draw the shape behind
  \fill[normal border]
        (A.south east) -- (A.south west) --
        (A.north west) -- (A.north east) -- cycle;
  \end{pgfonlayer}}}

% Macro to draw the shape, when the text will continue in next page
\def\parchmentframetop#1{
\tikz{
  \node[inner sep=2em] (A) {#1};    % Draw the text of the node
  \begin{pgfonlayer}{background}
  \fill[normal border]              % Draw the ``complete shape'' behind
        (A.south east) -- (A.south west) --
        (A.north west) -- (A.north east) -- cycle;
  \fill[torn border]                % Add the torn lower border
        ($(A.south east)-(0,.2)$) -- ($(A.south west)-(0,.2)$) --
        ($(A.south west)+(0,.2)$) -- ($(A.south east)+(0,.2)$) -- cycle;
  \end{pgfonlayer}}}

% Macro to draw the shape, when the text continues from previous page
\def\parchmentframebottom#1{
\tikz{
  \node[inner sep=2em] (A) {#1};   % Draw the text of the node
  \begin{pgfonlayer}{background}
  \fill[normal border]             % Draw the ``complete shape'' behind
        (A.south east) -- (A.south west) --
        (A.north west) -- (A.north east) -- cycle;
  \fill[torn border]               % Add the torn upper border
        ($(A.north east)-(0,.2)$) -- ($(A.north west)-(0,.2)$) --
        ($(A.north west)+(0,.2)$) -- ($(A.north east)+(0,.2)$) -- cycle;
  \end{pgfonlayer}}}

% Macro to draw the shape, when both the text continues from previous page
% and it will continue in next page
\def\parchmentframemiddle#1{
\tikz{
  \node[inner sep=2em] (A) {#1};   % Draw the text of the node
  \begin{pgfonlayer}{background}
  \fill[normal border]             % Draw the ``complete shape'' behind
        (A.south east) -- (A.south west) --
        (A.north west) -- (A.north east) -- cycle;
  \fill[torn border]               % Add the torn lower border
        ($(A.south east)-(0,.2)$) -- ($(A.south west)-(0,.2)$) --
        ($(A.south west)+(0,.2)$) -- ($(A.south east)+(0,.2)$) -- cycle;
  \fill[torn border]               % Add the torn upper border
        ($(A.north east)-(0,.2)$) -- ($(A.north west)-(0,.2)$) --
        ($(A.north west)+(0,.2)$) -- ($(A.north east)+(0,.2)$) -- cycle;
  \end{pgfonlayer}}}

% Define the environment which puts the frame
% In this case, the environment also accepts an argument with an optional
% title (which defaults to ``Example'', which is typeset in a box overlaid
% on the top border
\newenvironment{parchment}[1][Example]{%
  \def\FrameCommand{\parchmentframe}%
  \def\FirstFrameCommand{\parchmentframetop}%
  \def\LastFrameCommand{\parchmentframebottom}%
  \def\MidFrameCommand{\parchmentframemiddle}%
  \vskip\baselineskip
  \MakeFramed {\FrameRestore}
  \noindent\tikz\node[inner sep=1ex, draw=black!20,fill=white,
          anchor=west, overlay] at (0em, 2em) {\sffamily#1};\par}%
{\endMakeFramed}
%%%%%%%%%%%%%%%%%%%% END %%%%%%%%%%%%%%%%%%%%%%%


% Side notes
\NewEnviron{formula}[1]{
	\begin{center}
		\begin{minipage}{3.5in}
			\begin{parchment}[#1]
				\centering
				\BODY
			\end{parchment}
		\end{minipage}
	\end{center}
}

\NewEnviron{marginNote}{
	\marginpar{
		\begin{mdframed}[backgroundcolor=colorParchment,linewidth=0pt]
			\footnotesize
			\BODY
		\end{mdframed}
	}
}

% Emphasis paragraph
\global\mdfdefinestyle{emphasisParagraphFrameStyle}{
	topline=false,
	bottomline=false,
	rightline=false,
	skipabove=\topsep,
	skipbelow=\topsep,
	linecolor=xkcdBottleGreen,
	linewidth=3pt,
}

\newenvironment{emphasisParagraph}{
	\begin{quote}
	\begin{mdframed}[style=emphasisParagraphFrameStyle]
	\em
}{
	\end{mdframed}
	\end{quote}
}
