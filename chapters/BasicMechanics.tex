\chapterimage{image/dice.jpg}
\chapterspaceabove{6.75cm}
\chapterspacebelow{11.25cm}


\chapter{Challenges}

\begin{emphasisParagraph}
	For basic challenges, the actor rolls a \term{core capability}.
	A \term{physiology} or \term{expertise} bonus is added.
	The result of the roll has to be greater than the \term{difficulty level (DL)} to succeed.
\end{emphasisParagraph}
\todo{Does the success level matter? Think on it.}

\section{The Basic Mechanic}

\begin{formula}{The Basic Game Mechanic}
	\Large
	Roll \termCore{Capability} + Add \termBonus{Bonus} \\ Against \termDifficultyLevel{Difficulty Level}
\end{formula}


For basic challenges, the actor rolls a \termCore{core capability}.
Add a \termBonus{bonus} \termBonus{species} or \termBonus{expertise}, whichever is higher.
The result of the roll has to be greater than the \termDifficultyLevel{difficulty level (DL)} to succeed.


\paragraph*{Ties}
If the result of a roll, with the bonus, is \emph{equal} to the \acronymDL,
then this is a tie.

\paragraph*{The Double One}

Rolling and interpreting a \term{double one} is an important
mechanic. It is used for balancing out otherwise powerful
characters, for simulating potentially dangerous powerful magic
that can backfire on the individual. Finally, to simulate
that we learn from our failures, it is used for character
progression.

If the roll has two 1's, this is a \term{double one}.
This is the GM's opportunity to introduce a misfortune to the game:
The actor, either the player or an NPC, may still succeed, thanks to their bonus.
However, circumstances outside their control or a tragic character flaw creates a misfortune.
This rule applies to both players and NPCs.

Interpret the \term{double one} as independent of whether the character
succeeds, fails, or ties. A \term{double one} is a misfortune that will
impact for the rest of the rest, and it is something happening not
because the character was not good enough, but because circumstances were
beyond their control.

One way to introduce extraordinary elements to game is to increase the dice pool.
For example, magical artifacts can add a third, or even a fourth die to the dice pool.
In the cases where three or more dice are rolled,
rolling 1 on any two dice is enough to trigger a misfortune.
This mechanic increases the probability of a misfortune
whenever a third die is added. This mechanic represents
a point of failure for powerful actors, whether they be
PC or NPCs. For players, this mechanic can create
opportunities to roleplay immense power that comes at a cost.
These can be tragic heroes. For NPCs, this helps the GM
to create adversaries that are effectively unbeatable,
however, keep creating circumstances that players can exploit.

In addition, if a double-one is rolled and both of the ones
come from the character's own dice (i.e. they are the first
two dice in the dice pool) then the character increases either
one of the two dice. This is the main levelling up mechanic of
the system.


\paragraph*{Guideline on Deciding the Difficulty Level}

Imagine a talented character (2d6 in their \term{core capability}) with 20 years of experience.
This character's roll is 2d6 + 12 without any additional bonuses.
This means that they are approximately 50\% likely to overcome a challenge with \acronymDL 18.
It also means that they can overcome a challenge with \acronymDL 24 with a lot of luck - only 1/36 of the time.\\

\marginpar{
	\footnotesize
	\begin{tabularx}{\marginparwidth}{rX}
\multicolumn{2}{c}{Difficulty Levels} \\
3 & Teenagers fail half the time \\
6 & A teenager might fail, but an adult should get it right. \\
9 & An adult should succeed almost all the time \\
12 & Professional adults get it right half the time \\
18 & A talented adult with years of experience will get this half the time \\
\end{tabularx}
}

Now consider a regular person with no \term{expertise}. This person will roll d6 + d4.
They cannot overcome a challenge with 12,
and they will fail at relatively basic challenges of \acronymDL 5 half of the time.

\section{Character Advancement / Levelling Up}

\label{subsec:character_advancement}

We learn from the challenges we face. Rolls of 1 from the \emph{first two dice} trigger
character advancement.

In a \term{double one}, if the $1$'s are coming from the first two dice in the pool, this
triggers an increase in the \emph{greater} of the two dice.

If the two first dice contain at least one $1$ in two consecutive rows, this triggers an increase
in the \emph{lower} of the two dice.

If roleplaying kids or teenagers, the character will have only one die. When this die is rolled
$1$ twice in a row, add another die to the dice pool of the \termCore{core capability}.
This is a moment of growth and can be roleplayed as a rite of passage.

If desired, the level up mechanism applies can apply to all actors, including NPCs.

\section{Complex Challenges}

\section{Teamwork: Cooperative Challenges}

% TODO: The "flying city ritual" goes here..

\section{Competition: Opposing Rolls}

\section{The Spread Mechnanic}

\section*{Credits, References \& Bibliography}

Chapter header image credits: \cite{dice_image}

\printbibliography[heading=none]
