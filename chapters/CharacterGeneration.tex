\newcommand{\listUniversalTraining}{
	\begin{itemize}[leftmargin=.5cm]
    	\item Lasso Throwing
		\item Swimming
	\end{itemize}
}

\newcommand{\listUniversalSkills}{
	\begin{itemize}[leftmargin=.5cm]
		\item Carpentry
		\item Drawing
		\item Embroidery
		\item Knitting
		\item Lockpicking
		\item Massage
		\item Musical Instrument (\ldots)
		\item Painting
		\item Roof Tiling
		\item Shoemaking
		\item Sense of Direction
		\item Singing
		\item Sleight of Hand
		\item Tailoring
  		\item Chiropractice
  		\item Knots
  		\item Traps
	\end{itemize}
}

\newcommand{\listUniversalTalents}{
	\begin{itemize}[leftmargin=.5cm]
		\item Acting
		\item Animal Handling (\ldots)\footnote{\footnoteMundaneAnimals}
		\item Assertion
		\item Deception
  		\item Eye for Color
		\item Joking
		\item Language (\ldots)
		\item Listening
		\item Skepticism
		\item Taurascatics
	\end{itemize}
}

\newcommand{\listUniversalKnowledge}{
	\begin{itemize}[leftmargin=.5cm]
		\item Algebra
		\item Anatomy (Human)
		\item Anatomy (\ldots)\footnote{\footnoteMundaneAnimals}
		\item Astronomy
		\item Baking
		\item Bookkeeping
		\item Boat Maintenance
		\item Brewing
		\item Cooking
		\item Ceramics
		\item Maritime Currents (\ldots)\footnote{Specify the ocean or sea.}
		\item Herbology
		\item Philosophy
		\item Physics
		\item Politics
		\item Theology (...)
		\item Winemaking
	\end{itemize}
}




\newcommand{\listModernTraining}{
	\begin{itemize}[leftmargin=.5cm]
		\item Bouldering
		\item Cycling
		\item Dancing (\ldots)\footnote{Ballroom, Capoeira, Hiphop, Latin, Tango, \ldots}
		\item Defensive Martial Art (\ldots)
		\item Gymnastics
		\item Knife Fighting
		\item Offensive Martial Art (\ldots)
		\item Wrestling
	\end{itemize}
}

\newcommand{\listModernSkills}{
	\begin{itemize}[leftmargin=.5cm]
		\item Driving (Cars)
		\item Riding (Motorcycle)
		\item Electrical Wiring
		\item Engine Repair (Gas)
	\end{itemize}
}

\newcommand{\listModernKnowledge}{
	\begin{itemize}[leftmargin=.5cm]
		\item Accounting
		\item Architecture
		\item Art History
		\item Astronomy
		\item Biology
		\item Calculus
		\item Cinema
		\item Civil Engineering
		\item Clinical Skills
		\item Computing Concepts
		\item Criminal Law (Anglosaxon)
		\item Economics\footnote{Macro, unless otherwise specified.}
		\item Electrical Engineering
		\item Electronics
		\item Firearms
		\item Firefighting
		\item First Aid
		\item Interior Design
		\item Mechanics
		\item Medical Equipment (\ldots)
		\item Music Theory
		\item Photography
		\item Physics
		\item Psychology
		\item Sociology
		\item Theatre
		\item World History
	\end{itemize}
}

\newcommand{\listModernTalents}{
	\begin{itemize}[leftmargin=.5cm]
		\item Fashion Sense
	\end{itemize}
}



\newcommand{\listMedievalSkills}{
	\begin{itemize}[leftmargin=.5cm]
		\item Blacksmithing
		\item Locksmithing
		\item Riding (Horse)
		\item Riding (Camel)
		\item Stonemasonry
		\item Wheelwrighting
		\item Tanning
	\end{itemize}
}

\newcommand{\listMedievalTraining}{
	\begin{itemize}[leftmargin=.5cm]
		\item Archery
		\item Maces
		\item Swords
		\item Nunchakus
		\item Off-hand Dagger Use
	\end{itemize}
}

\newcommand{\listMedievalKnowledge}{
	\begin{itemize}[leftmargin=.5cm]
		\item History (Local)
	\end{itemize}
}

\newcommand{\listFantasySkills}{
	\begin{itemize}[leftmargin=.5cm]
		\item Clairvoyance
		\item Mediumship
		\item Telekinesis
		\item Pyrokinesis
	\end{itemize}
}

\newcommand{\listFantasyTalents}{
	\begin{itemize}[leftmargin=.5cm]
		\item
	\end{itemize}
}

\newcommand{\listFantasyTraining}{
	\begin{itemize}[leftmargin=.5cm]
		\item Animal Handling (...)\footnote{Gigantic birds, gigantic arachnids, etc\ldots}
		\item Riding (Griffin)
		\item Riding (Dragon)
	\end{itemize}
}

\newcommand{\listFantasyKnowledge}{
	\begin{itemize}[leftmargin=.5cm]
		\item Alchemy
		\item Astrology
		\item Curses \& Hexes
		\item Demonology
		\item Necromancy
		\item Oneiromancy
		\item Summoning Rituals
		\item Tarot
		\item Thaumaturgy
	\end{itemize}
}

\chapterimage{image/baby_faces.jpg}
\chapterspaceabove{2.75cm}
\chapterspacebelow{5.25cm}


\chapter{Character Creation}

\section{Create a \term{Mundane} Character}
\begin{marginNote}
	\term{Mundane} characters are humans in a modern setting.
	Spells, powers \& overpowered \termBeyond{core capabilities} take the character \term{beyond the mundane}.
\end{marginNote}
Creating a character is like writing a social network profile or a CV, but more truthful.
You thought about their background, so reflect your thoughts as numbers:
\begin{enumerate}
	\item Assign their \termCore{core capabilities}.
	\item The character gets \termBonus{species bonuses} - "human" for \term{mundane} characters.
	\item Because they are sentient, the character also gets bonuses for their \termBonus{areas of expertise}, based on their years of experience.
\end{enumerate}

Characters roll their \termCore{core capabilities} and
add their \termBonus{bonuses} from their species, either \term{physiology} or their \term{expertise}, whichever is higher.
The bonus is limited by the maximum possible dice roll, for example, if the \termCore{capability} rolle is $2d6$, the \termBonus{bonus} can be 12 at most.

\paragraph*{1. Core Capabilities}

\begin{marginNote}
	\begin{tabular}{rl}
d4 & Child \\
d6 & Teenager \\
2d4 & Not your strong suit \\
d4 + d6 & An average adult \\
2d6 & This is your strong suit \\
\end{tabular}
\end{marginNote}
Assign $2d4$ if they are bad,
$2d6$ if they are good, and $d4 + d6$ if they are average.
For weak characters, you can use $d6$ - inexperienced teenagers, people with disadvantages or injuries, or debilitatings sicknesses.
Larger dice ($d8$, $d10$, \ldots) are for going \termBeyond{beyond the mundane}, for supernatural characters.

% TODO: Formalize this a bit more to keep mundane characters balanced.

\paragraph*{2. Species Bonuses}
As a human, the character gets +7 to vision\footnote{
	To calculate the vision for another species, divide their visual acuity, cycles-per-degree by ten.
	For example, an eagle has 140 cyc/deg, so their bonus is +14.
	See \cite{caves_acuityview_2018} for more information on visual acuity,
	and \cite{noauthor_list_2019} for a list of visual acuities of animals.
},
and +2 to hearing.\footnote{
	To calculate the hearing for another species, divide their audible range in Hz by 10000.
	For example, bats can hear up to 100kHz, so a bat character will have +10.
}
In addition, they get +1 or +2 as \termBonus{upper body strength}.\footnote{
	The average human punch has pressure of 150 psi, we divide that by 100 to get the bonus.
	Some humans are stronger than others, due to age, gender, nutrition differencces.
	If needed, based on their background, gauge how much pressure their punch will pack in psi, and divide by 100.
	For trained humans, the \termBonus{expertise bonus} will likely be higher.
}
See \nameref{subsec:chargen_mundane} for more details on these stats.

% TODO: Consider adding a kick.

% TODO: Explain strength.

% Examples include the smell of dogs, infrared vision, ultraviolet vision,
% magnetoreceptivity of bees and some birds, and electroreceptivity of sharks.

\paragraph*{3. Expertise}

\begin{marginNote}
	\begin{tabular}{rl}
Years & Bonus \\
\hline
1 yr & +1 \\
2 yrs & +2 \\
3 yrs & +3 \\
\multicolumn{2}{c}{...} \\
9 yrs & +9 \\
10 yrs & +10 \\
11 yrs & +10 \\
12 yrs & +10 \\
\multicolumn{2}{c}{...} \\
19 yrs & +10 \\
20 yrs & +11 \\
21 yrs & +11 \\
\multicolumn{2}{c}{...} \\
30 yrs & +12 \\
\end{tabular}
\end{marginNote}
These are bonuses related to a character's history \& experience up to date.
Assign these bonuses based on the character's background:
Bonus equals the years of experience up to the tenth year.
After the tenth year, count each decade as one point.
If they work on it only part-time, count as half-year.\par
\termBonus{Training} is pure muscle memory, you learn these by training regularly.
\termBonus{Skills} are things that require concentration to accomplish. Think of these as skills for blue-collared jobs or hobbies.
\termBonus{Knowledge} are pure theoretical knowledge, 100-level university courses or night classes.
\termBonus{Talents} are social.
The final bonuses need to make sense to \emph{all} other players when you are introducing your character.\par

Below is a sample character based roughly on yours truly.
Hover over the \hover{highlighted scores}{This is a tooltip.} to see a tooltip that explains how each of those were calculated.\footnote{
	The PDF viewer that you are using should be showing \hover{this as highlighted}{This is a tooltip.} with a message that says "this is a toolip".
	Otherwise, you will need another PDF viewer to see the tooltips.
}




% Explain again: Knowledge is everything you either take a course for, or apprentice to learn.
% Cannot be too specific: night-time classes or 100-level courses.
% Cannot be too broad: High school classes and things that everyone knows are too broad.
% Training is pure muscle memory, and is physically exhausting.
% Skills are things that require focus. You typically learn these in a night class, or you might apprentice.
% Talents are things that you develop by doing. They can be by nature or by nurture.

\newcommand{\explainEE}{The character took 4 years of classes related to Electrical Engineering.This gives +4 bonus under \termBonus{knowledge}.  His work between the ages 24 and 26 involved components used in electrical engineering. This sums up to 6 years, or a bonus of +6.}
\newcommand{\explainEcon}{The character took multiple classes during the master's degree. I decided that this sums up 2 or 3, and went with the lower number.}
\newcommand{\explainFinance}{Two years of classes on finance and no work experience.}
\newcommand{\explainCycling}{The character can cycle: they taught themselves as a kid. For about six summers, they spent a lot of summers cycling. However, it's just the summer, and not the full year. I divide 6 years in half, and get +3 as the bonus.}
\newcommand{\explainDriving}{Driving is something that you would take lessons to learn. Furthermore, it is a mixture of theoretical knowledge and training. So it is classified as a \term{skill}. The character has been driving regularly since the age of 22, so the bonus is +5.}
\newcommand{\explainProgramming}{He started programming at the age of 9, did that as a hobby pretty much every year until the age of 22. I decide to divide this by half - it's not equivalent of full time work. However, between the ages 22 and 26, he worked full time programming a system, so I count that as 4. The total is 10.}
\newcommand{\explainSkepticism}{When the character started work, they found themselves in situations where they need to understand when people are being deceptive. This is the only social talent that this character has. While difficult to quantify in terms of years of experience, the player decides that this is fair based on the character history \& concept.}


\begin{actorCardLetterSizeFitToPage}{testCharacter}{Sample Mundane Character}{colorBgMundane}

	\begin{capabilitiesBox}{testCharacter}
		\begin{capabilitiesTable}{Physical}
			\capability{Coordination}{2d4}
			\capability{Constitution}{d4+d6}
		\end{capabilitiesTable}
		\begin{capabilitiesTable}{Mental}
 			\capability{Situational Awareness}{d4+d6}
 			\capability{Focus}{2d6}
		\end{capabilitiesTable}
		\begin{capabilitiesTable}{Intellectual}
 			\capability{Reasoning}{2d6}
 			\capability{Willpower}{2d4}
		\end{capabilitiesTable}
		\begin{capabilitiesTable}{Social}
			\capability{Communication}{d4+d6}
			\capability{Social Awareness}{2d4}
	   \end{capabilitiesTable}
	\end{capabilitiesBox}

	\begin{speciesBonusBox}{testCharacter}
		\begin{bonusTable}{Physical}
			\bonus{Core}{+1}
			\bonus{Upper Body}{+1}
		\end{bonusTable}
		\begin{bonusTable}{Sensory}
			\bonus{Vision (Daylight)}{+7}
			\bonus{Hearing}{+1}
		\end{bonusTable}
	\end{speciesBonusBox}

	\expertiseTable{testCharacter}{
        \expertise{Cycling}{\hover{+3}{\explainCycling}}
        \expertise{Dancing (Latin)}{+3}
        \expertise{Swimming}{+3}
	}{
        \expertise{Driving (Car)}{\hover{+3}{\explainDriving}}
	}{
        \expertise{Accounting}{+1}
        \expertise{Chemistry}{+1}
        \expertise{Computing Concepts}{\hover{+10}{\explainProgramming}}
        \expertise{Economics}{\hover{+2}{\explainEcon}}
        \expertise{Electrical Engineering}{\hover{+6}{\explainEE}}
        \expertise{Finance}{\hover{+2}{\explainFinance}}
        \expertise{Marketing}{+1}
        \expertise{Physics}{+1}
	}{
        \expertise{Language (English)}{+12}
        \expertise{Language (French)}{+2}
        \expertise{Skepticism}{\hover{+4}{\explainSkepticism}}
	}

\end{actorCardLetterSizeFitToPage}

\begin{formula}{The Basic Game Mechanic}
	\Large
	Roll \termCore{Capability} + Add \termBonus{Bonus} \\ Against \termDifficultyLevel{Difficulty Level}
\end{formula}

% TODO: Add the list of core capabilities somewhere in this chapter as a full list.
% TODO: Add a table of dice rolls and their meanings.




\section{Play the Character}

\subsection{Mundane Challenges}

% TODO: The 4th and 5th are competitions, remove them, or change the DL.

\begin{center}
	\rowcolors{1}{white}{xkcdEggShell}
	\begin{xltabular}{\textwidth}{Xlll}
		\textbf{Situation}																& \textbf{Roll} 														& \textbf{Dice}	& \textbf{DL}	\\
		Find a solution to an engineering problem at work.								& \termCore{Reasoning} + \termBonus{Electrical Engineering}				& $2d6+6$ 		& 12 \\
		Meet someone at a party.														& \termCore{Social Awareness}											& $2d4$ 		& 9 \\
		Respond to a fire that just started												& \termCore{Situational Awareness} + \termBonus{Firefighting} 			& $d4+d6+0$ 	& 12 \\
		Notice a hidden door by checking the wall dimensions							& \termCore{Situational Awareness} + \termBonus{Architecture} 			& $2d6+0$ 		& 12 \\
																						& \termCore{Situational Awareness} + \termBonus{Stonemasonry} 			& $d4+d6+0$ 	& 12 \\
		See through deception during conversation										& \termCore{Reasoning} + \termBonus{Skepticism}							& $d4+d6+5$ 	& \hover{---}{Opposed Roll: \termCore{Communication} + \termBonus{Deception}} \\
		Deceive someone during conversation 											& \termCore{Communication} + \termBonus{Deception}						& $d4+d6+0$ 	& \hover{---}{Opposed Roll: \termCore{Reasoning} + \termBonus{Skepticism}} \\
		Throw a fist at an unsuspecting individual										& \termCore{Coordination} + \termBonus{Strength}						& $2d4+1$ 		& 6 \\
	\end{xltabular}
\end{center}

\subsection*{Beyond the Mundane}

\begin{center}
	\rowcolors{1}{white}{xkcdEggShell}
	\begin{xltabular}{\textwidth}{Xll}
		Situation												& Challenge 								& Roll \\
		They need to identify some potions through sight, smell \& taste. 				& \termCore{Situational Awareness} + Herbology		& d4 + d6 + 0 \\
	\end{xltabular}
\end{center}




% \footnote{
% 	If worked on something more than half of a year, that's a full year, round up. If less than a year (for instance, summers only) divide the total number of years by half.
% }

\section{Areas of Expertise}


\termBonus{Training} is physical.
You can have a trainer, but what matters is that you train day in, day out.
For example, if they went \termBonus{Bouldering} every weekend or so for two years, count each one of those as a half year, because it is part time, and assign +1.
Alternatively, if the character taught a form of \termBonus{offensive martial arts} for five years, assign +5 as the bonus.\par
\termBonus{Knowledge} is pure theory.
In the modern setting, these are 100-level university courses, or sometimes, night classes.
Consider all years studying in the university to calculate the bonus, as well working in a job which actively requires the knowledge in question.
In other settings, also consider the years fully apprenticing.
When adding more areas of \termBonus{knowledge} to the campaign, consider the following:
If you go to 200-level or 300-level courses, it is too specific for the game mechanic.
For settings that are \termBonus{beyond the mundane}, consider the highest level of
education that will prepare the character for a high paying job, or a respectable role in the society that requires specific knowledge.
This can be schooling in magic, or it can be apprenticing scholars, or working with elders.
\par
\termBonus{Talents} are social, and cannot be easily taught, characters get better at these by doing.\par

If something can be taught with less then one hour of instructions, that is not an area of \term{expertise}.
This is something that an adult should be able to do very easily after hearing the instructions,
so the \term{difficulty level} of using such an equipment will likely be 3 on simple checks,
and should be determined only by the \term{core capability}.
This includes crossbox and modern firearms.
\footnote{"Firearms", listed under knowledge, is about encyclopedic knowledge of firearms and their maintenance, not for shooting.}


% \section*{Resources: Passive Income \& Renown}

% House
% Social Support
% Work Network
% Criminal Network?



% \marginpar{
% 	\vspace{7cm}
% 	\footnotesize
% 	Any income to the chraracter that does not involve a job is \term{passive income}. This can be rent from a real estate, parental support, or from investments. (Exact score determination needs to be decided, this can be a dice or a bonus.)\\

% 	\term{Renown} is the based on the number of you people who know of the character, without the character knowing them.
% 	To determine the score, get the number of people who would know of the chracter, and count the number of digits.
% 	For example, a researcher well known in their field could be 2 or even 3.
% 	An modern-era influencer could go up to 4 (1000 followers or more), 5, (10000 followers or more).
% 	A celebrity could be at 6 or 7, or even 8 if they are very famous.
% 	Most people should have 0 or, perhaps 1 if they have an online presence.
% }


\pagebreak
\subsection{Mundane Areas of Expertise}

\begin{center}
	\scriptsize
	\begin{xltabular}{\textwidth}{XXXXl}
		Training				&	Skills					& Knowledge					&  Talents 				& \\
		\hline
		\listUniversalTraining	&	\listUniversalSkills	&	\listUniversalKnowledge	& \listUniversalTalents & \sideTab{colorBgMundane}{All Settings} \\
		\listModernTraining		&	\listModernSkills		&	\listModernKnowledge	& \listModernTalents	& \sideTab{colorBgModern}{Modern Setting} \\
		\listMedievalTraining	&	\listMedievalSkills		&	\listMedievalKnowledge	&   					& \sideTab{colorBgMedieval}{Medieval Setting} \\
	\end{xltabular}
\end{center}

\subsection{Beyond the Mundane}

\begin{center}
	\begin{xltabular}{\textwidth}{XXXXl}
		Training								&	Skills									& Knowledge								&  Talents & \\
		\hline
		\footnotesize \listFantasyTraining		&	\footnotesize	\listFantasySkills		&\footnotesize \listFantasyKnowledge	& \footnotesize  \listFantasyTalents 	& \sideTab{colorBgBeyondMundane}{Beyond the Mundane} \\
	\end{xltabular}
\end{center}


\section{Characters of Other Species}

When creating an actor, you have to at least list their senses and their physical strength in physical altercation.
For a list of animals, see the section
% TODO: Make it into a book and complete this paragraph.

\subsection[Mundane Species]{Mundane Species}
\label{subsec:chargen_mundane}

Earlier, we wrote that humans get +7 to their \termBonus{vision}.
This is calculated based on something called visual acuity.
To calculate the vision for another species, divide their visual acuity, cycles-per-degree by ten.
For example, an eagle has 140 cyc/deg, so their bonus is +14.
See \cite{caves_acuityview_2018} for more information on visual acuity,
and \cite{noauthor_list_2019} for a list of visual acuities of animals.

% TODO: Add a list of visual acuities of animals.



\section{Put Characters in Various Campaigns}
\subsection{Mundane}
\begin{center}
	\rowcolors{1}{white}{xkcdEggShell}
	\begin{xltabular}{\textwidth}{XX}
		Campaign & Character Creation \\
		\hline
		A group of high school friends come back to their home town when one of them passes away. As they mourn their friends' passing, they begin to suspect foulplay and a coverup... & Create characters as usual. \\
		A band of volunteer soldiers find themselves in a war. & Create characters as usual. \\
		In a longer campaign, a group of firegihters respond to crises in a particularly flammable \& dangerous city. & Create characters as usual. Obviously, the background should reflect their years in the firefighting force. \\
	\end{xltabular}
\end{center}


\subsection{Beyond the Mundane}
\begin{center}
	\rowcolors{1}{white}{xkcdEggShell}
	\begin{xltabular}{\textwidth}{XX}
		Campaign & Character Creation \\
		\hline
		A band of talented teenagers investigate paranormal phenomenon.
			& Create teenagers with d6 in all \termCore{core capabilities}, except for their one special talent, which is 2d8.\\
   		A tightly-knit group of kids are attending a secret school of magic.
			& Create kids with d4 in all \termCore{core capabilities}. However, also add in the fantasy knowledge and talents. Add a spellbook. \\
		An artist has a deal with a fairy to get a 3rd die on their \termCore{Focus} rolls for creating art.
			Based on a roll or card draw, either the fairy disappears, or offers a 4th die, or turns out to be a Deep One.
			& Create a \term{mundane} character as on top of the book. \\
			% TODO: Finish the character creation when the character sheet is finished.
			% TODO: Link to the powers of fey
	\end{xltabular}
\end{center}

\section{Balance Gameplay for Characters}

With this level of flexibility, finding the right difficulty for the challenges may be difficult. Look at the players' rolls, and find the maximum roll possible. For example, for an young professional,
this number is likely going to be 18: This is going to be a roll with a \term{core capability} where they are particulary talented, so this is going to be 2d6. The highest possible roll is 12. They probably have at least five or six years of experience in their job, so this brings us up to 18. This is the character's \term{mundane level}.

Plan a scene around one climatic challenge at this \term{difficulty level}, potentially for another \term{core capability} and another area of \term{expertise}. They should have a very low or zero probability of passing. However, using item bonuses, their \term{resources}, and planning, they should be able to bring this to a high probability event. There will still be the probability of a double-one, a \term{misfortune}. Even if they are very well prepared when they face the challenge, if they roll double ones, they may still pass, but a misfortune falls upon them.

\section*{Credits, References \& Bibliography}

Chapter header image credits: \cite{baby_faces}

\printbibliography[heading=none]
