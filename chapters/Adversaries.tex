\chapterimage{image/squirrel.jpg}
\chapterspaceabove{6.75cm}
\chapterspacebelow{11.25cm}


\chapter{Adversaries \& Companions}

\begin{emphasisParagraph}
	The mechanics for PCs and NPCs are the same: If an NPC has a power or capability,
	a PC can also have it.
\end{emphasisParagraph}

\sbox{0}{
	\begin{actorCardMiniEuro}{mundaneHazardFire}{Fire}{colorBgMundane}
		\begin{capabilitiesBox}{mundaneHazardFire}
			\begin{capabilitiesTable}{Physical}
				\capability{Burn}{1d4}
			\end{capabilitiesTable}
		\end{capabilitiesBox}
	\end{actorCardMiniEuro}
}

\begin{wrapfigure}{r}{\wd0}
	\usebox{0}
\end{wrapfigure}


In movies and literature, the characters face conflicts.
Conflicts can be internal, or external.
If external, they can come from other characters, or from nature.
In the game mechanic, all of these adversaries are \term{actors}, as well as the players.
NPCs are \term{actors}: they roll dice and add bonuses, just as PCs do.
Human characters are just one type of \term{actor} and follow the same mechanics.
All \term{actors} can be roleplayed by players, or introduced as NPCs to the game.

Non-human phenomena are also actors: a fire, a sickness, and even inner demons are all
sources of conflict. The more predictable ones only have bonuses, and the more
chaotic, unpredictable phenomena like fire have dice rolls.
In short, if a phenomenon can be anthropomorphized, it is an \term{actor}.

\begin{emphasisParagraph}
	If a phenomenon can be anthropomorphized, it is an \term{actor}.
\end{emphasisParagraph}



\section{Hazards}

\subsection{Fire}

Fire is a monster and an \term{actor}: it has no \termCore{reasoning} or \termCore{situational awareness},
but it has the capability to burn.


Each round, fire rolls its \termCore{Burn}. Every time it rolls the face die (4 at first),
it spreads into an adjacent tile.
Once it goes over 4 tiles in either direction, its \termCore{Burn} increases to $d6$.
The maximum number of tiles in each direction determines one of the two dice in the \termCore{Strength} roll, rounded up.
\todo{Add an image using tikz.}\\

\todo{Describe how to douze the fire.}
\todo{Describe what happens if the fire rolls a double-one.}

\todo{Add a rule that says that the heating capability of the fire decreases as we go away from the sides.}

\todo{Link to the fire elemental subsection}

\sbox{0}{
	\begin{actorCardMiniAmerican}{mundaneHazardInferno}{Raging Inferno}{colorBgMundane}
		\begin{capabilitiesBox}{mundaneHazardInferno}
			\begin{capabilitiesTable}{Physical}
				\capability{Burn}{2d20}
			\end{capabilitiesTable}
		\end{capabilitiesBox}
	\end{actorCardMiniAmerican}
}


% TODO: Add "choke" and make a master list of capabilities.

\begin{wrapfigure}{l}{\wd0}
	\usebox{0}
\end{wrapfigure}


\subsection{Shock / Electricity}

% TODO: This is going to be just one bonus, and it needs to be calculated from the power surge.



\section{Sickness}

\section{Inner Demons}

\section{Injuries}

\section{Animals \& Creatures}
\label{sec:animals}

\subsection{Small Mammals}

% TODO: label this section, and reference using the

Small mammals such as squirrels use \termCore{Situational Awareness} to recognize food and danger,
and \termCore{Coordination} to avoid them.
Unlike domesticated pets and familiars, they cannot communicate with
humans, so they have no \termCore{Communication}.
Please do not hurt them.\\

% TODO: Find out more about the vision, smelling and senses of squirrels and put it here.
\sbox{0}{
	\begin{actorCardMiniEuro}{squirrel}{Squirrel}{colorBgMundane}
		\begin{capabilitiesBox}{squirrel}
			\begin{capabilitiesTable}{Capabilities}
				\capability{Coordination}{2d4}
				\capability{Situational Awareness}{d4+d6}
			\end{capabilitiesTable}
		\end{capabilitiesBox}

		\begin{speciesBonusBox}{squirrel}
			\begin{bonusTable}{Sensory}
				\bonus{Vision (Daylight)}{+7}
				\bonus{Hearing}{+1}
			\end{bonusTable}
		\end{speciesBonusBox}
	\end{actorCardMiniEuro}
}
% TODO: Speed?
\todo{Add skills: gather nuts, evade}
\todo{Add physiology: size & bite strength, speed}
\todo{Add how to show a squirrel with human intelligence}

\begin{wrapfigure}{r}{\wd0}
	\usebox{0}
\end{wrapfigure}



\subsection{A Dog}
\label{subsec:dog}
Dogs have the following four core competencies at 1d4.
Unlike many other animals, they can communicate with humans.
Their bite attacks are \termCore{Coordination} + Jaw Strength\\
% \begin{characterSheet}
% 	\begin{tabular}{lr|clr|}
% 		\multicolumn{2}{c}{Mental}	&&	\multicolumn{2}{c}{Physical} \\
% 		\cline{1-2} \cline{4-5}
% 		Situational Awareness & 1d4				&& 	Coordination & 1d4 \\
% 		\multicolumn{4}{c}{} \\
% 		\multicolumn{2}{c}{Social}	&&	\multicolumn{2}{c}{Innate} \\
% 		\cline{1-2} \cline{4-5}
% 		Communication & 1d4 			&&	Focus & 1d4 \\
% 	\end{tabular}
% \end{characterSheet}
\todo{Add skills: fetch, open doors, swim}
\todo{Add physiology: size & bite strength}

\todo{Add a wolf. Also add a werewolf to beyond the mundane.}

\section{Beyond the Mundane}

\subsection{Fire Elemental}
\label{subsec:fire_elemental}
\todo{Write the Fire Elemental. It needs to have Reasoning as a core competence.}

\section*{Credits, References \& Bibliography}

Chapter header image credits: \cite{squirrel_playing_chess}

\printbibliography[heading=none]
