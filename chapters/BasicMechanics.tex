\chapterimage{image/dice.jpg}
\chapterspaceabove{6.75cm}
\chapterspacebelow{11.25cm}


\chapter{Challenges}

\begin{emphasisParagraph}
	For basic challenges, the actor rolls a \term{core capability}.
	A \term{physiology} or \term{expertise} bonus is added.
	The result of the roll has to be greater than the \term{difficulty level (DL)} to succeed.
\end{emphasisParagraph}
\todo{Does the success level matter? Think on it.}

\section{The Basic Mechanic}

\begin{formula}{The Basic Game Mechanic}
	\Large
	Roll \termCore{Dice} + Add \termBonus{Bonus} \\ Against \termDifficultyLevel{Difficulty Level}
\end{formula}


For basic challenges, the actor rolls a \termCore{core capability}.
Add a \termBonus{bonus} \termBonus{species} or \termBonus{expertise}, whichever is higher.
The result of the roll has to be greater than the \termDifficultyLevel{difficulty level (DL)} to succeed.

\paragraph{The Dice \& The Bonus}

First two dice are determined by the actor's \termCore{core capabilities}.
(See Section \nameref{sec:character_creation}.) The exception to having two dice
is when playing teenagers --- teenagers will have one die only in their
\termCore{core capabilities}, and the ``earning'' of a new dice can be
roleplayed as a rite of passage --- See \nameref{subsec:character_advancement}.


In addition to the dice from the actor's \termCore{core capabilities},
magical items, extraordinary situations can \emph{add} more dice to the pool.
These dice are for representing different types of extraordinary elements, such
as inspiration bonus from an extraordinary speech or a magical bonus coming
from being a vampire or werewolf. Mechanically, these dice increase the
probability of \emph{both a success, and suffering a misfortune} --- see below for the
``\term{double one}'' mechanic. The purpose
of this mechanic is to be able to create powerful actors, either as PC or NPCs that
are also vulnerable due to various misfortunes.

\begin{emphasisParagraph}
	The \termCore{dice} stack and add up, while only the highest \termBonus{bonus} is used.
	Increasing the number \termCore{dice} increases the probability of success, but
	also increases the probability of a misfortune. For \termBonus{bonuses},
	items and circumstances are important for actors with limited \termBonus{expertise},
	and make no difference for actors with high levels of \termBonus{expertise}.
\end{emphasisParagraph}


\termBonus{Bonuses} do not stack.
They can come from either \termBonus{expertise} of the character,
non-magical or technological items, or from circumstances.

For example, consider the situation where we roll to find out if
an actor can determine if the seeds of a particular plant are
poisonous for a particular animal or beast. The character may have a
bonus from their expertise --- see the Section \nameref{sec:character_creation}.
Or, the character may be receiving a bonus by using a cell phone connected to the Internet.
% TODO: Add a link here when the Items chapter is finished.
Alternatively, they can invest time in research in a library to learn more
--- see Section \nameref{subsec:research_mechanic}.




\paragraph*{Ties}
If the result of a roll, with the bonus, is \emph{equal} to the \acronymDL,
then this is a tie.

\paragraph*{The Double One}

Rolling and interpreting a \term{double one} is an important
mechanic. It is used for balancing out otherwise powerful
characters, for simulating potentially dangerous powerful magic
that can backfire on the individual. Finally, to simulate
that we learn from our failures, it is used for character
progression.

If the roll has two 1's, this is a \term{double one}.
This is the GM's opportunity to introduce a misfortune to the game:
The actor, either the player or an NPC, may still succeed, thanks to their bonus.
However, circumstances outside their control or a tragic character flaw creates a misfortune.
This rule applies to both players and NPCs.

Interpret the \term{double one} as independent of whether the character
succeeds, fails, or ties. A \term{double one} is a misfortune that will
impact for the rest of the rest, and it is something happening not
because the character was not good enough, but because circumstances were
beyond their control.

One way to introduce extraordinary elements to game is to increase the dice pool.
For example, magical artifacts can add a third, or even a fourth die to the dice pool.
In the cases where three or more dice are rolled,
rolling 1 on any two dice is enough to trigger a misfortune.
This mechanic increases the probability of a misfortune
whenever a third die is added. This mechanic represents
a point of failure for powerful actors, whether they be
PC or NPCs. For players, this mechanic can create
opportunities to roleplay immense power that comes at a cost.
These can be tragic heroes. For NPCs, this helps the GM
to create adversaries that are effectively unbeatable,
however, keep creating circumstances that players can exploit.

In addition, if a double-one is rolled and both of the ones
come from the character's own dice (i.e. they are the first
two dice in the dice pool) then the character increases either
one of the two dice. This is the main levelling up mechanic of
the system.


\paragraph*{Guideline on Deciding the Difficulty Level}

Imagine a talented character (2d6 in their \term{core capability}) with 20 years of experience.
This character's roll is 2d6 + 12 without any additional bonuses.
This means that they are approximately 50\% likely to overcome a challenge with \acronymDL 18.
It also means that they can overcome a challenge with \acronymDL 24 with a lot of luck - only 1/36 of the time.\\

\marginpar{
	\footnotesize
	\begin{tabularx}{\marginparwidth}{rX}
\multicolumn{2}{c}{Difficulty Levels} \\
3 & Teenagers fail half the time \\
6 & A teenager might fail, but an adult should get it right. \\
9 & An adult should succeed almost all the time \\
12 & Professional adults get it right half the time \\
18 & A talented adult with years of experience will get this half the time \\
\end{tabularx}
}

Now consider a regular person with no \term{expertise}. This person will roll d6 + d4.
They cannot overcome a challenge with 12,
and they will fail at relatively basic challenges of \acronymDL 5 half of the time.

\section{Character Advancement / Levelling Up}

\label{subsec:character_advancement}

The first two of the dice that actors roll depend on them.
Character advancement is repserented by increasing these dice.
As these two dice increase, more difficult tasks get within reach of the actor,
and the probability of a \term{double one}s decreases.

Just like in real life, actors learn from their failures and hardship.
Rolls of 1 from the \emph{first two dice} trigger character advancement.\footnote{Other dice, such as the dice that are added through magical items or conditions do not count for character advancement.}

\paragraph*{Increasing the higher of the two dice}
In a \term{double one}, if the $1$'s are coming from the first two dice in the pool, this
triggers an increase in the \emph{greater} of the two dice.

If roleplaying kids or teenagers, the character will have only one die. When this die is rolled
$1$ twice in a row, add another die to the dice pool of the \termCore{core capability}.
This is a moment of growth and can be roleplayed as a rite of passage.

If desired, the level up mechanism applies can apply to all actors, including NPCs.

\section{Complex Challenges}

\section{Teamwork: Cooperative Challenges}

% TODO: The "flying city ritual" goes here..

\section{Competition: Opposing Rolls}

\section{The Spread Mechnanic}

\section{Preparation \& Research Mechanics}

``Preparation'' is a way to give actors a way to get a \termBonus{bonus}
even in cases where they lack the bonus from \termBonus{expertise}.
\term{Research} is a good example to start.

\subsection{Research}
\label{subsec:research_mechanic}

\begin{marginNote}
	\begin{tabularx}{\marginparwidth}{rX}
\multicolumn{2}{c}{Duration Table} \\
\hline
+0 & Instantaneous or one second \\
+1 & 1 minute \\
+2 & 10 minutes \\
+3 & 1 hour \\
+4 & 1 day \\
+5 & 1 week \\
+6 & 1 month \\
+7 & 1 year \\
+8 & 1 decade \\
+9 & 1 century \\
+10 & 1 millenium \\
+11 & 10 millenia \\
\end{tabularx}
\end{marginNote}

Consider the situations where we roll to find out if an actor knows the answer
to a particular question. For example, in an ordinary setting, we might need
to roll to see if the actor knows the name of a director of a local charity.
In fantasy settings, they might need to roll to see if they know the name of
a particular demon. These situation call for knowledge roll, typically
\termCore{Reasoning} + \termBonus{a specific area of knowledge}.

In these cases, decleare \emph{ahead of the roll} if you want to do research and how
much time to spend on the research. Based on the time spent, the actor gets a \termBonus{bonus}
to their roll. If this \termBonus{bonus} exceeds other bonuses, then they use this
\termBonus{bonus}.

Whether research is possible or not depends on the context, setting, and equipment.
A party can engage in research if they are near a library in a fantasy setting.
However, a party near the entrance to a cave of hostile creatures is no longer able to do research.
Conversely, a party in a modern setting will be able to do research using their cell phones. (See Section \nameref{sec:items_table})
However, a party in a modern setting in a subterranean cave will not be able to use their cell phones because there is no reception.


\subsection{Other Preparation Mechanics}

\term{Preparation} mechanic applies to other preparation situations as well.
For instance, consider aiming with a firearm: if a player declares that they are taking aim,
simply by spending one minute, they are able to get a +1 bonus.
This bonus is irrelevant if they have an expertise bonus: since the bonus from expertise
will be higher than the preparation bonus, we will use the expertise bonus.
This reflects real life: aiming would become second nature to somebody who used firearms
in their line of work for over a couple of years, and they will be aiming far more accurately,
even without thinking about it.

\section*{Credits, References \& Bibliography}

Chapter header image credits: \cite{dice_image}

\printbibliography[heading=none]
