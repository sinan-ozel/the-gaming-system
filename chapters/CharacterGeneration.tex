\newcommand{\listUniversalTraining}{
	\begin{itemize}[leftmargin=.5cm]
		\item Hunting \footnote{\footnoteMundaneAnimals}
    	\item Lasso Throwing
		\item Swimming
	\end{itemize}
}

\newcommand{\listUniversalSkills}{
	\begin{itemize}[leftmargin=.5cm]
		\item Carpentry
		\item Drawing
		\item Embroidery
		\item Fishing, Line \footnote(Sea, Lake, River)
		\item Gathering \footnote{Arctic, Tropical, Boreal, Grasslands, Swamps, Desert, Highlands, Subterranean, Urban}
		\item Knitting
		\item Lockpicking
		\item Massage
		\item Musical Instrument (\ldots)
		\item Painting
		\item Roof Tiling
		\item Shoemaking
		\item Sense of Direction
		\item Singing
		\item Sleight of Hand
		\item Tailoring
  		\item Chiropractice
  		\item Knots
  		\item Traps
	\end{itemize}
}

\newcommand{\listUniversalTalents}{
	\begin{itemize}[leftmargin=.5cm]
		\item Acting
		\item Animal Handling (\ldots)\footnote{\footnoteMundaneAnimals}
		\item Assertion
		\item Deception
  		\item Eye for Color
		\item Joking
		\item Language (\ldots)
		\item Listening
		\item Skepticism
		\item Taurascatics
	\end{itemize}
}

\newcommand{\listUniversalKnowledge}{
	\begin{itemize}[leftmargin=.5cm]
		\item Algebra
		\item Anatomy (Human)
		\item Anatomy (\ldots)\footnote{\footnoteMundaneAnimals}
		\item Astronomy
		\item Baking
		\item Bookkeeping
		\item Boat Maintenance
		\item Brewing
		\item Cooking
		\item Ceramics
		\item Maritime Currents (\ldots)\footnote{Specify the ocean or sea.}
		\item Herbology
		\item Philosophy
		\item Physics
		\item Politics
		\item Theology (...)
		\item Winemaking
	\end{itemize}
}




\newcommand{\listModernTraining}{
	\begin{itemize}[leftmargin=.5cm]
		\item Bouldering
		\item Cycling
		\item Dancing (\ldots)\footnote{Ballroom, Capoeira, Hiphop, Latin, Tango, \ldots}
		\item Defensive Martial Art (\ldots)
		\item Gymnastics
		\item Knife Fighting
		\item Offensive Martial Art (\ldots)
		\item Wrestling
	\end{itemize}
}

\newcommand{\listModernSkills}{
	\begin{itemize}[leftmargin=.5cm]
		\item Driving (Cars)
		\item Riding (Motorcycle)
		\item Electrical Wiring
		\item Engine Repair (Gas)
	\end{itemize}
}

\newcommand{\listModernKnowledge}{
	\begin{itemize}[leftmargin=.5cm]
		\item Accounting
		\item Architecture
		\item Art History
		\item Biology
		\item Calculus
		\item Cinema
		\item Civil Engineering
		\item Clinical Skills
		\item Computing Concepts
		\item Criminal Law (Anglosaxon)
		\item Economics\footnote{Macro, unless otherwise specified.}
		\item Electrical Engineering
		\item Electronics
		\item Firearms
		\item Firefighting
		\item First Aid
		\item Interior Design
		\item Mechanics
		\item Medical Equipment (\ldots)
		\item Music Theory
		\item Photography
		\item Physics
		\item Psychology
		\item Sociology
		\item Theatre
		\item World History
	\end{itemize}
}

\newcommand{\listModernTalents}{
	\begin{itemize}[leftmargin=.5cm]
		\item Fashion Sense
	\end{itemize}
}



\newcommand{\listMedievalSkills}{
	\begin{itemize}[leftmargin=.5cm]
		\item Blacksmithing
		\item Locksmithing
		\item Riding (Horse)
		\item Riding (Camel)
		\item Stonemasonry
		\item Wheelwrighting
		\item Tanning
	\end{itemize}
}

\newcommand{\listMedievalTraining}{
	\begin{itemize}[leftmargin=.5cm]
		\item Archery
		\item Maces
		\item Swords
		\item Nunchakus
		\item Off-hand Dagger Use
	\end{itemize}
}

\newcommand{\listMedievalKnowledge}{
	\begin{itemize}[leftmargin=.5cm]
		\item History (Local)
	\end{itemize}
}

\newcommand{\listFantasySkills}{
	\begin{itemize}[leftmargin=.5cm]
		\item Clairvoyance
		\item Mediumship
		\item Telekinesis
		\item Pyrokinesis
	\end{itemize}
}

\newcommand{\listFantasyTalents}{
	\begin{itemize}[leftmargin=.5cm]
		\item
	\end{itemize}
}

\newcommand{\listFantasyTraining}{
	\begin{itemize}[leftmargin=.5cm]
		\item Animal Handling (...)\footnote{Gigantic birds, gigantic arachnids, etc\ldots}
		\item Riding (Griffin)
		\item Riding (Dragon)
	\end{itemize}
}

\newcommand{\listFantasyKnowledge}{
	\begin{itemize}[leftmargin=.5cm]
		\item Alchemy
		\item Astrology
		\item Curses \& Hexes
		\item Demonology
		\item Necromancy
		\item Oneiromancy
		\item Summoning Rituals
		\item Tarot
		\item Thaumaturgy
	\end{itemize}
}

\chapterimage{image/baby_faces.jpg}
\chapterspaceabove{2.75cm}
\chapterspacebelow{5.25cm}


\chapter{Character Creation}

\section{Create an \term{Ordinary} Character}

\label{sec:character_creation}

\begin{marginNote}
	\term{Ordinary} characters are humans in a modern setting.
	Increasing the \term{dice pool} in \termCore{core capabilities}
	is one way of creating \termBeyond{extraordinary} characters.
\end{marginNote}
To create the character, decide on their life story so far,
then we reflect than in dice and bonuses.
\begin{enumerate}
	\item Based on their childhood \& teenage years, assign their \termCore{core capabilities}.
	\item Based on their years of experience, assign their \termBonus{areas of expertise}.
\end{enumerate}

As a player, you will roll the dice from the \termCore{core capabilities},
and add a bonus from the \termBonus{areas of expertise}.
If the sum is higher than the \termDifficultyLevel{difficulty level} of a task,
your character succeeds at that task.

\paragraph*{1. Core Capabilities}

\begin{marginNote}
	\textbf{TL;DR for Capabilities}
	For each core capability,
	add either $d4$ or $d6$ from childhood,
	add another $d4$ or $d6$ from teenage years.
\end{marginNote}

Choose the capabilities to represent the experiences of the character's childhood and teenage years.
For each one of the 8 core capabilities in the table below, choose two dice.
First die comes from the childhood. If they had a childhood which would
make them exceptional in this \termCore{core capability}, use $d6$, otherwise, it is a $d4$.
If you want to have a prodigy in the area, use $d8$.
Repeat for the teenage years. You will wind up with
two dice for each \termCore{core capability}.

\begin{marginNote}
	You do not have to fully finish character generation to start playing.
	You can play a flashback scene to the character's childhood or teenage
	years when you first need to roll a \termCore{core capability} or
	an \termBonus{area of expertise}.
\end{marginNote}

\begin{emphasisParagraph}
The \termCore{core capabilities} need to make sense to \emph{all} other players
when you are introducing your character.
If you made them raise their eyebrows or roll their eyes,
character creation failed.
\end{emphasisParagraph}


\setval{widthCapability = 29.75mm}
\begin{capabilitiesTable}{Athletic}
	\capability{Coordination}{2d4}
	\capability{Constitution}{d4+d6}
\end{capabilitiesTable}
\begin{capabilitiesTable}{Mental}
	\capability{Situational Awareness}{d4+d6}
	\capability{Focus / Creativity}{2d6}
\end{capabilitiesTable}
\begin{capabilitiesTable}{Intellectual}
	\capability{Reasoning}{2d6}
	\capability{Willpower}{2d4}
\end{capabilitiesTable}
\begin{capabilitiesTable}{Social}
	\capability{Communication}{d4+d6}
	\capability{Social Awareness}{2d4}
\end{capabilitiesTable}

\paragraph*{2. Expertise}

\begin{marginNote}
	\begin{tabular}{rl}
Years & Bonus \\
\hline
1 yr & +1 \\
2 yrs & +2 \\
3 yrs & +3 \\
\multicolumn{2}{c}{...} \\
9 yrs & +9 \\
10 yrs & +10 \\
11 yrs & +10 \\
12 yrs & +10 \\
\multicolumn{2}{c}{...} \\
19 yrs & +10 \\
20 yrs & +11 \\
21 yrs & +11 \\
\multicolumn{2}{c}{...} \\
30 yrs & +12 \\
\end{tabular}
\end{marginNote}
Choose the bonuses the same way you talk about work expertise on a CV.
How many years did the character spend as a professional or in higher education?
This number of years is the bonus of all \termBonus{areas of expertise}.
See Section \nameref{subsec:areas_of_expertise} for the full list.

The final \termBonus{bonuses} need to make sense to \emph{all} other players
when you are introducing your character.\par



\newcommand{\explainEE}{The character took 4 years of classes related to Electrical Engineering.This gives +4 bonus under \termBonus{knowledge}.  His work between the ages 24 and 26 involved components used in electrical engineering. This sums up to 6 years, or a bonus of +6.}
\newcommand{\explainEcon}{The character took multiple classes during the master's degree. I decided that this sums up 2 or 3, and went with the lower number.}
\newcommand{\explainFinance}{Two years of classes on finance and no work experience.}
\newcommand{\explainCycling}{The character can cycle: they taught themselves as a kid. For about six summers, they spent a lot of summers cycling. However, it's just the summer, and not the full year. I divide 6 years in half, and get +3 as the bonus.}
\newcommand{\explainDriving}{Driving is something that you would take lessons to learn. Furthermore, it is a mixture of theoretical knowledge and training. So it is classified as a \term{skill}. The character has been driving regularly since the age of 22, so the bonus is +5.}
\newcommand{\explainProgramming}{He started programming at the age of 9, did that as a hobby pretty much every year until the age of 22. I decide to divide this by half - it's not equivalent of full time work. However, between the ages 22 and 26, he worked full time programming a system, so I count that as 4. The total is 10.}
\newcommand{\explainSkepticism}{When the character started work, they found themselves in situations where they need to understand when people are being deceptive. This is the only social talent that this character has. While difficult to quantify in terms of years of experience, the player decides that this is fair based on the character history \& concept.}



\setval{widthExpertise = 25mm}
\expertiseTable{
	\expertise{Cycling}{\hover{+3}{\explainCycling}}
	\expertise{Dancing (Latin)}{+3}
}{
	\expertise{Driving (Car)}{\hover{+3}{\explainDriving}}
	\expertise{Musical Instrument (Guitar)}{\hover{+5}{\explainDriving}}
}{
	\expertise{Computing Concepts}{\hover{+10}{\explainProgramming}}
	\expertise{Economics}{\hover{+2}{\explainEcon}}
}{
	\expertise{Language (English)}{+12}
	\expertise{Language (French)}{+2}
}


\clearpage
\begin{tikzpicture}[overlay]
	\node[draw, fill=LightYellow, thick,minimum width=4cm,text width=4cm,minimum height=3cm] (b) at (-3.5,-3.5
	){
		Characters get one die each from their childhood and teenage years.
		If there is a reason that they would develop a capability early on, change
		the die to $d6$. Otherwise, it is $d4$. Use $d8$ only if they are a prodigy,
		a truly exceptional person.
	};
\end{tikzpicture}

\begin{tikzpicture}[overlay]
	\node[draw, fill=LightYellow, thick,minimum width=4cm,text width=4cm,minimum height=3cm] (b) at (-3.5,-7.0
	){
		The bonus is equal to the characters' years of experience working to sustain themselves.
		This includes years spent as an adult on a primary occupation, apprenticeship,
		and/or college education.
	};
\end{tikzpicture}

\begin{actorCardLetterSizeFitToPage}{testCharacter}{Sample Ordinary Adult}{colorBgMundane}

	\begin{capabilitiesBox}{testCharacter}
		\begin{capabilitiesTable}{Athletic}
			\capability{Coordination}{2d4}
			\capability{Constitution}{d4+d6}
		\end{capabilitiesTable}
		\begin{capabilitiesTable}{Mental}
 			\capability{Situational Awareness}{d4+d6}
 			\capability{Focus / Creativity}{2d6}
		\end{capabilitiesTable}
		\begin{capabilitiesTable}{Intellectual}
 			\capability{Reasoning}{2d6}
 			\capability{Willpower}{2d4}
		\end{capabilitiesTable}
		\begin{capabilitiesTable}{Social}
			\capability{Communication}{d4+d6}
			\capability{Social Awareness}{2d4}
	   \end{capabilitiesTable}
	\end{capabilitiesBox}

	\begin{expertiseBox}{testCharacter}
		\expertiseTable{
			\expertise{Cycling}{\hover{+3}{\explainCycling}}
			\expertise{Dancing (Latin)}{+3}
			\expertise{Swimming}{+3}
		}{
			\expertise{Driving (Car)}{\hover{+3}{\explainDriving}}
			\expertise{Musical Instrument (Guitar)}{\hover{+5}{\explainDriving}}
		}{
			\expertise{Accounting}{+1}
			\expertise{Chemistry}{+1}
			\expertise{Computing Concepts}{\hover{+10}{\explainProgramming}}
			\expertise{Economics}{\hover{+2}{\explainEcon}}
			\expertise{Electrical Engineering}{\hover{+6}{\explainEE}}
			\expertise{Finance}{\hover{+2}{\explainFinance}}
			\expertise{Marketing}{+1}
			\expertise{Physics}{+1}
		}{
			\expertise{Language (English)}{+12}
			\expertise{Language (French)}{+2}
			\expertise{Skepticism}{\hover{+4}{\explainSkepticism}}
		}
	\end{expertiseBox}


	\begin{physiologyBox}{testCharacter}
		\begin{tabularx}{\textwidth}{X}
			This space is reserved for physilogy mechanics which replace the hit point
			mechanic. (Robustness, natural armor)
		\end{tabularx}
	\end{physiologyBox}

\end{actorCardLetterSizeFitToPage}

\begin{formula}{The Basic Game Mechanic}
	\Large
	Roll \termCore{Capability} + Add \termBonus{Bonus} \\ Compare to \termDifficultyLevel{Difficulty Level}
\end{formula}

% TODO: Add the list of core capabilities somewhere in this chapter as a full list.
% TODO: Add a table of dice rolls and their meanings.




\section{Play the Character}

\subsection{Ordinary Challenges}

% TODO: The 4th and 5th are competitions, remove them, or change the DL.

\begin{center}
	\rowcolors{1}{white}{xkcdEggShell}
	\begin{xltabular}{\textwidth}{Xlll}
		\textbf{Situation}																& \textbf{Roll} 														& \textbf{Dice}	& \textbf{DL}	\\
		Find a solution to an engineering problem at work.								& \termCore{Reasoning} + \termBonus{Electrical Engineering}				& $2d6+6$ 		& 12 \\
		Meet someone at a party.														& \termCore{Social Awareness}											& $2d4$ 		& 9 \\
		Respond to a fire that just started												& \termCore{Situational Awareness} + \termBonus{Firefighting} 			& $d4+d6+0$ 	& 12 \\
		Notice a hidden door by checking the wall dimensions							& \termCore{Situational Awareness} + \termBonus{Architecture} 			& $2d6+0$ 		& 12 \\
																						& \termCore{Situational Awareness} + \termBonus{Stonemasonry} 			& $d4+d6+0$ 	& 12 \\
		See through deception during conversation										& \termCore{Reasoning} + \termBonus{Skepticism}							& $d4+d6+5$ 	& \hover{---}{Opposed Roll: \termCore{Communication} + \termBonus{Deception}} \\
		Deceive someone during conversation 											& \termCore{Communication} + \termBonus{Deception}						& $d4+d6+0$ 	& \hover{---}{Opposed Roll: \termCore{Reasoning} + \termBonus{Skepticism}} \\
		Throw a fist at an unsuspecting individual										& \termCore{Coordination} + \termBonus{Strength}						& $2d4+1$ 		& 6 \\
	\end{xltabular}
\end{center}

\subsection*{Extraordinary}

\begin{center}
	\rowcolors{1}{white}{xkcdEggShell}
	\begin{xltabular}{\textwidth}{Xll}
		\textbf{Situation}													& \textbf{Challenge} 								& \textbf{Roll} \\
		They need to identify some potions through sight, smell \& taste. 	& \termCore{Situational Awareness} + Herbology		& d4 + d6 + 0 \\
	\end{xltabular}
\end{center}





\section{Areas of Expertise}

\label{subsec:areas_of_expertise}

How many years did the character spend working to sustain themselves and in training?
This years of experience is represented as a bonus added to the roll, up until the tenth year.
Full-time work, college education are included.
(For instance, if they took economics classes in one year in college, that one year counts towards the bonus.
If they worked in a job requiring economics knowledge, those years also count toward the bonus.)

If some of this is done as a hobby, before adulthood, or as a part-time job, include those as half-years.
After ten years, each decade counts as +1. For \termBeyond{characters that go beyond normal lifetimes},
after a century, each century counts as +1. See the side table.

\subsection{Creating More Areas of Expertise}

When adding new areas of expertise, to keep bonuses consistent,
authors or DMs can use the principles used to create the table above:

\termBonus{Training} is pure muscle memory, you learn these by training regularly.
You can have a trainer, but what matters is that you train day in, day out.
For example, if they went \termBonus{Bouldering} every weekend or so for two years, count each one of those as a half year, because it is part time, and assign +1.
Alternatively, if the character taught a form of \termBonus{offensive martial arts} for five years, assign +5 as the bonus.\par

\termBonus{Skills} are things that require concentration to accomplish.
In a modern setting, think of these as skills needed for blue-collared jobs, vocations or hobbies.

\termBonus{Knowledge} is pure theory.
In the modern setting, these are 100-level university courses, or sometimes, night classes.
Consider all years studying in the university to calculate the bonus, as well working in a job which actively requires the knowledge in question.
In other settings, also consider the years fully apprenticing.
When adding more areas of \termBonus{knowledge} to the campaign, consider the following:
If you go to 200-level or 300-level courses, it is too specific for the game mechanic.
For fantasy and other settings, consider the highest level of
education that will prepare the character for a high paying job, or a respectable role in the society that requires specific knowledge.
This can be schooling in magic, or it can be apprenticing scholars, or working with elders.
\par
\termBonus{Talents} are social, and cannot be easily taught, characters get better at these by doing.\par

If something can be taught with less then one hour of instructions, that is not an area of \term{expertise}.
This is something that an adult should be able to do very easily after hearing the instructions,
so the \term{difficulty level} of using such an equipment will likely be 3 on simple checks,
and should be determined only by the \term{core capability}.
This includes crossbox and modern firearms.
\footnote{"Firearms", listed under knowledge, is about encyclopedic knowledge of firearms and their maintenance, not for shooting.}


% \section*{Resources: Passive Income \& Renown}

% House
% Social Support
% Work Network
% Criminal Network?



% \marginpar{
% 	\vspace{7cm}
% 	\footnotesize
% 	Any income to the chraracter that does not involve a job is \term{passive income}. This can be rent from a real estate, parental support, or from investments. (Exact score determination needs to be decided, this can be a dice or a bonus.)\\

% 	\term{Renown} is the based on the number of you people who know of the character, without the character knowing them.
% 	To determine the score, get the number of people who would know of the chracter, and count the number of digits.
% 	For example, a researcher well known in their field could be 2 or even 3.
% 	An modern-era influencer could go up to 4 (1000 followers or more), 5, (10000 followers or more).
% 	A celebrity could be at 6 or 7, or even 8 if they are very famous.
% 	Most people should have 0 or, perhaps 1 if they have an online presence.
% }


\subsection{Ordinary Areas of Expertise}

\begin{center}
	\scriptsize
	\begin{xltabular}{\textwidth}{XXXXl}
		Training				&	Skills					& Knowledge					&  Talents 				& \\
		\hline
		\listUniversalTraining	&	\listUniversalSkills	&	\listUniversalKnowledge	& \listUniversalTalents & \sideTab{colorBgMundane}{All Settings} \\
		\listModernTraining		&	\listModernSkills		&	\listModernKnowledge	& \listModernTalents	& \sideTab{colorBgModern}{Modern Setting} \\
		\listMedievalTraining	&	\listMedievalSkills		&	\listMedievalKnowledge	&   					& \sideTab{colorBgMedieval}{Medieval Setting} \\
	\end{xltabular}
\end{center}

\subsection{Extraordinary Expertise}

\begin{center}
	\begin{xltabular}{\textwidth}{XXXXl}
		Training								&	Skills									& Knowledge								&  Talents & \\
		\hline
		\footnotesize \listFantasyTraining		&	\footnotesize	\listFantasySkills		&\footnotesize \listFantasyKnowledge	& \footnotesize  \listFantasyTalents 	& \sideTab{colorBgBeyondMundane}{Beyond the Mundane} \\
	\end{xltabular}
\end{center}


\section{Advance the Character}

Rules for advancing --- ``levelling up'' --- the character are in the
Section~\nameref{subsec:character_advancement}.


\section{Place Characters in Campaigns}
\subsection{Ordinary}
\begin{center}
	\rowcolors{1}{white}{xkcdEggShell}
	\begin{xltabular}{\textwidth}{XX}
		Campaign & Character Creation \\
		\hline
		A group of high school friends come back to their home town when one of them passes away. As they mourn their friends' passing, they begin to suspect foulplay and a coverup... & Create characters as usual. \\
		A band of volunteer soldiers find themselves in a war. & Create characters as usual. \\
		In a longer campaign, a group of firegihters respond to crises in a particularly flammable \& dangerous city. & Create characters as usual. Obviously, the background should reflect their years in the firefighting force. \\
	\end{xltabular}
\end{center}


\subsection{Extraordinary}
\begin{center}
	\rowcolors{1}{white}{xkcdEggShell}
	\begin{xltabular}{\textwidth}{XX}
		Campaign & Character Creation \\
		\hline
		A band of talented teenagers investigate paranormal phenomenon.
			& Create teenagers with d6 in all \termCore{core capabilities}, except for their one special talent, which is 2d8.\\
   		A tightly-knit group of kids are attending a secret school of magic.
			& Create kids with d4 in all \termCore{core capabilities}. However, also add in the fantasy knowledge and talents. Add a spellbook. \\
		An artist has a deal with a fairy to get a 3rd die on their \termCore{Focus} rolls for creating art.
			Based on a roll or card draw, either the fairy disappears, or offers a 4th die, or turns out to be a Deep One.
			& Create a \term{mundane} character as on top of the book. \\
			% TODO: Finish the character creation when the character sheet is finished.
			% TODO: Link to the powers of fey
	\end{xltabular}
\end{center}

\section{Balance Gameplay for Characters}

With this level of flexibility, finding the right difficulty for the challenges may be difficult. Look at the players' rolls, and find the maximum roll possible. For example, for an young professional,
this number is likely going to be 18: This is going to be a roll with a \term{core capability} where they are particulary talented, so this is going to be 2d6. The highest possible roll is 12. They probably have at least five or six years of experience in their job, so this brings us up to 18. This is the character's \term{mundane level}.

Plan a scene around one climatic challenge at this \term{difficulty level}, potentially for another \term{core capability} and another area of \term{expertise}. They should have a very low or zero probability of passing. However, using item bonuses, their \term{resources}, and planning, they should be able to bring this to a high probability event. There will still be the probability of a double-one, a \term{misfortune}. Even if they are very well prepared when they face the challenge, if they roll double ones, they may still pass, but a misfortune falls upon them.

\section*{Credits, References \& Bibliography}

Chapter header image credits: \cite{baby_faces}

\printbibliography[heading=none]
