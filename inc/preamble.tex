% Add some TODO notes for the writer's own use.
\usepackage[disable]{todonotes}

% This package combines longtable and tabularx
\usepackage{xltabular}

% Add the highlighted notes that appear when the reader hovers
\usepackage{acro}
\usepackage{pdfcomment}

% Get some nice pre-defined colors.
% Start here: https://ctan.math.ca/tex-archive/macros/latex/contrib/xkcdcolors/xkcdcolors-manual.pdf
% See the git page: https://github.com/Rmano/xkcdcolors
% See the full list here: https://xkcd.com/color/rgb/
% See here for the story: https://blog.xkcd.com/2010/05/03/color-survey-results/
\usepackage{xkcdcolors}

% Define coloring related to terms
% I am using "Contrasting Palette 1 & 2" from https://venngage.com/tools/accessible-color-palette-generator
\definecolor{ocre}{RGB}{196, 70, 1} % Define the color used for highlighting throughout the boo
\definecolor{colorCoreCompetency}{RGB}{91, 163, 0}

\definecolor{colorCoreCompetency}{RGB}{91, 163, 0}
\newcommand{\termCore}[1]{\textcolor{colorCoreCompetency}{#1}}

\definecolor{colorTerm}{RGB}{196, 70, 1}
\newcommand{\term}[1]{\textcolor{colorTerm}{#1}}

\newcommand{\termClass}[1]{\textcolor{xkcdSpearmint}{#1}}

% Add the nice side tabs on the tables.
% Usage: Put into the rightmost cell in a tabularx environment.
% Example: \sideTab{LightBlue}{Modern}
\newcommand{\sideTab}[2]{\cellcolor{#1} \rotatebox[origin=l]{270}{#2}}

% Add color to table cells
% See here: https://tex.stackexchange.com/questions/50349/color-only-a-cell-of-a-table
\usepackage{colortbl}

% Use this package to break up a list into multiple columns
\usepackage{multicol}

% Add styling to emphasized paragraphs and character sheets
\usepackage{mdframed}

\global\mdfdefinestyle{d20ComparisonBoxStyle}{
	linewidth=1pt,
	linecolor=xkcdSandy,
	frametitlebackgroundcolor=xkcdPale,
	backgroundcolor=xkcdPale,
	frametitlefont=\bfseries\sffamily\footnotesize,
	frametitlerule=true,
}


\global\mdfdefinestyle{emphasisParagraphFrameStyle}{
	topline=false,
	bottomline=false,
	rightline=false,
	skipabove=\topsep,
	skipbelow=\topsep,
	linecolor=xkcdBottleGreen,
	linewidth=3pt,
}

\newenvironment{emphasisParagraph}{
	\begin{quote}
	\begin{mdframed}[style=emphasisParagraphFrameStyle]
	\em
}{
	\end{mdframed}
	\end{quote}
}




% Style the bullets in itemized lists
% https://tex.stackexchange.com/questions/42805/what-are-original-itemize-bullet-definitions
\renewcommand\labelitemi{\textbullet}
\renewcommand\labelitemii{\normalfont\bfseries \textendash}
\renewcommand\labelitemiii{\textasteriskcentered}
\renewcommand\labelitemiv{\textperiodcentered}


% TODO: Remove this.
\newcommand{\coreCompetencyTable}[8]{
	\begin{center}
	\resizebox{\columnwidth}{!}{
		\begin{tabular}{lr|clr|}
			\multicolumn{2}{c}{Mental}	&&	\multicolumn{2}{c}{Physical} \\
			\cline{1-2} \cline{4-5}
			Reasoning & #1				&& 	Coordination & #3 \\
			Situational Awareness & #2	&& 	Constitution & #4 \\
			\multicolumn{4}{c}{} \\
			\multicolumn{2}{c}{Social}	&&	\multicolumn{2}{c}{Innate} \\
			\cline{1-2} \cline{4-5}
			Communication & #5 			&&	Focus & #7 \\
			Social Awareness & #6 		&&	Creativity & #8 	\\
		\end{tabular}
	}
	\end{center}
}

\newcommand{\hover}[2]{
	\pdfmarkupcomment[markup=Highlight,disable=false,color=LightYellow]{#1}{#2}
}

\newcommand{\acronym}[2]{
	\pdfmarkupcomment[markup=Underline,disable=false,color=colorTerm]{#1}{#2}
}

\newcommand{\acronymDL}[1]{
	\pdfmarkupcomment[markup=Underline,disable=false,color=colorTerm]{DL}{
		Difficulty Level: A roll has to beat this level number to succeed.
	}
}


% TODO: Remove the following
\newenvironment{characterSheet}{
	\begin{minipage}{\columnwidth}
	\begin{mdframed}
	\begin{center}
}
{
	\end{center}
	\end{mdframed}
	\end{minipage}
}

\newcommand{\expertiseTableOld}[3]{
	\begin{tabular}{ccc}
		\multicolumn{3}{c}{} \\
		Knowledge & Skills & Talents \\
		\hline
		\begin{tabular}[t]{lr|}
			#1
		\end{tabular} &
		\begin{tabular}[t]{lr|}
			#2
		\end{tabular} &
		\begin{tabular}[t]{lr}
			#3
		\end{tabular} \\
	\end{tabular}
}

% TODO: Remove when done.
\newcommand{\physiologyTableOld}[3]{
	\begin{tabular}{ccc}
		\multicolumn{3}{c}{} \\
		Senses & Strength & Movement \\
		\hline
		\begin{tabular}[t]{lr|}
			#1
		\end{tabular} &
		\begin{tabular}[t]{lr|}
			#2
		\end{tabular} &
		\begin{tabular}[t]{lr}
			#3
		\end{tabular} \\
	\end{tabular}
}

\usepackage{wrapfig}

% Add tikzpicture to use for sheets
\usepackage{tikz}
\usetikzlibrary{positioning, calc, shadings}

% Font for handwriting
\usepackage{emerald}

\usepackage{makecell}

\newenvironment{actorSheet}{
	\begin{tikzpicture}
		\draw[color=xkcdLightPlum,line width=2mm,rounded corners=.55cm]
			 	(0,0) node (SW) {} --
				(0,5.5in) node (NW) {} --
				(4.25in,5.5in) node (NE) {} --
				(4.25in,0) node (SE) {} -- (current subpath start)
			;
}{
	\end{tikzpicture}
}


\newcommand{\labeledFrame}[4]{
	\node[draw=xkcdLightPlum,top color=xkcdLightPink,middle color=xkcdVeryLightPink,bottom color=white,font=\normalsize,#3,anchor=north west,line width=.5mm,inner ysep=1mm,inner xsep=2mm] (#1){
		#4
	};
	\node[text=xkcdVeryLightPink,fill=xkcdLightPlum,node font=\sffamily,below right=0.2mm and -0.1mm of #1.north west,anchor=south west] (#1Label)  {
		\hspace*{5mm}#2
	};
	\coordinate[below right=.2mm and -.1mm of #1Label.north east,anchor=west] (#1A);
	\coordinate[right=-.1mm of #1Label.south east,anchor=west] (#1B);
	\coordinate[left=.6mm of #1.north east,anchor=south west] (#1C);
	\draw[xkcdLightPlum,fill=xkcdLightPlum] (#1A) -- (#1B) -- (#1C) -- (#1A);
}


\newcommand{\capabilityTable}[2]{
	\labeledFrame{tabCapability}{\footnotesize #1}{below right=.35 in and .15 in of NW}{
		\begin{tabular}{lr}
			#2
		\end{tabular}
	}
}

\newcommand{\physiologyTable}[4]{
	\node[draw=xkcdMutedBlue,fill=xkcdReallyLightBlue,top color=xkcdReallyLightBlue,middle color=xkcdReallyLightBlue,bottom color=white,#1,font=\normalsize,anchor=north west,line width=.5mm,inner ysep=0mm,inner xsep=0mm] (tabPhysiology){
		\begin{tabular}[t]{c|c|c}
			\rowcolor{xkcdMutedBlue}
			\sffamily\footnotesize\textcolor{xkcdIvory}{Strength}	& \sffamily\footnotesize\textcolor{xkcdIvory}{Senses}	& \sffamily\footnotesize\textcolor{xkcdIvory}{Natural Armor} \\
			\begin{tabular}[t]{p{1.6cm}r}#2\end{tabular}			& \begin{tabular}[t]{p{1.6cm}r}#3\end{tabular} 			& \begin{tabular}[t]{p{1.6cm}r}#4\end{tabular}
		\end{tabular}
	};
}

\newcommand{\expertiseTable}[4]{
	\node[draw=xkcdMutedBlue,fill=xkcdReallyLightBlue,top color=xkcdReallyLightBlue,middle color=xkcdReallyLightBlue,bottom color=white,#1,font=\normalsize,anchor=north west,line width=.5mm,inner ysep=0mm,inner xsep=0mm] (tabExpertise){
		\begin{tabular}[t]{c|c|c}
			\rowcolor{xkcdMutedBlue}
			\sffamily\footnotesize\textcolor{xkcdIvory}{Knowledge}	& \sffamily\footnotesize\textcolor{xkcdIvory}{Skills}	& \sffamily\footnotesize\textcolor{xkcdIvory}{Talents} \\
			\begin{tabular}[t]{p{1.5cm}r}#2\end{tabular}			& \begin{tabular}[t]{p{1.5cm}r}#3\end{tabular} 			& \begin{tabular}[t]{p{1.5cm}r}#4\end{tabular}
		\end{tabular}
	};
}





\newcommand{\coreCapability}[2]{
	\\[-3mm]
	#1 & \ECFAugie\footnotesize #2 \\
	\hline
}

\newcommand{\bonusLine}[2]{
	#1 & \ECFAugie\footnotesize #2 \\
}

\newcommand{\expertiseLine}[2]{
	\ECFAugie\footnotesize #1 & \ECFAugie\footnotesize #2 \\
}

%TODO: Add fields on top for: Name, Species, Size
%TODO: Add movement
%TODO: Add trauma scale
%TODO: Injuries, Mental States
